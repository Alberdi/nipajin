% (c) 2009-2015 by Markus Leupold-Löwenthal
% This file is released under CC BY-SA 4.0.

\renewcommand{\dwarvesVersion}{v0.1}

% CHANGELOG-de
%
% 0.1
%   - first translation, based on v1.0.1-de

% --- language dependent typography stuff --------------------------------------

\renewcommand{\fsNormal}{\fontsize{9.25pt}{11pt plus 0.1pt minus 0pt}}
\renewcommand{\fsSmall}{\fontsize{8pt}{9.5pt plus 0.1pt minus 0pt}}

% --- pdf metadata -------------------------------------------------------------

\hypersetup{
	pdfsubject={A short NIP'AJIN scenario featuring wet dwarves.},
}

% --- fine print ---------------------------------------------------------------

\renewcommand{\dwarvesCredits}{nobody}

% --- main texts ---------------------------------------------------------------

\renewcommand{\dwarvesPlayers}{%
	A scenario for three to six characters.
}
\renewcommand{\dwarvesHeadline}{Wet Dwarves}
\renewcommand{\dwarvesToc}{Scenario: Wet Dwarves}
\renewcommand{\dwarvesText}{%
	{\itshape
		Deep below a snowy mountain chain in the center of the world there live the dwarves -- small and a bit chubby humanoids with round faces, big noses, and long beards. Only a few of them ever have seen daylight, as their passion are the caves and tunnels, the stones and rocks, the gems and ores of this world.

		You are dwarves and members of the Quartz clan, who found happiness in mining white, rose or yellow crystals. In Tvurud, the cozy cave you call home, extended over hundreds of years into a huge hall, everything glitters. Reflecting pillars and few torches and lanterns are all that is needed to bring light in every corner. As members of the elite of Tvurud you usually do not go down into the mines yourself. Instead, you patrol the vicinity and protect your clan from hostile neighbors. Cave monsters, goblins, and dark elves are only a few of the dangers that lurk in the darkness. And from time to time find their way into your home.

		While enjoying a few hours of peace and quietness in Tvurdu, the hypnotic \say{Tik-Di-Tik} of dozens of stonemasons is disrupted by a loud rumbling noise. \say{Break-in! Water! Our pals need help!} someone shouts. You quickly grab your gear and run into the mine. All hands are busy to help friends and co-workers. The stressed foreman cries, \say{Three of us are still missing down there!} and directs you into a tunnel, where the water is already ankle-deep.
	}

	\mysection{Setting}

		\noindent
		This scenario is set in a dwarven mine. The dwarves meet the typical fantasy cliché. The mine consist of a network of tunnels supported by wooden beams. Rails run through most of them. There is glitter everywhere -- pebble-sized gems can be found in almost any wall, crystals in larger caves sometimes are twice the height of a dwarf. In regular intervals lanterns are set up against the darkness and their reflections carry the light even deeper into the tunnels.

	\mysection{Characters}

		\noindent
		All players personify dwarves from the following table. Each PC is afraid of water (Swimming-4) and gets two more hindrances -1 assigned by his player. Each dwarf also gets a special item, \eg\ a weapon that does 2 instead of 1 damage, armor (+1 on \RD\ rolls), or another item that rewards +1 on certain rolls. Every PC also has a lantern, as dwarves in this setting can not see in the dark.

		\tabelle{l c X}{
		\thead{PC} & \thead{\HD} & \thead{Skills} \\
		}{
			Digger       &  8 & Mining+1, Pickaxe+1 \\
			Fighter      & 10 & Axe+2 \\
			Priest       &  6 & Power:Icy Touch+1, Power:Heal \\
			Sharpshooter &  6 & Crossbow+2 \\
			Thief        &  6 & Stealth+1, Agility+1 \\
			Wizard       &  4 & two powers of choice \\
		}

		\noindent
		Priests and Wizards can use each of their powers five times.

	\mysection{Development}

		\noindent
		The break-in was triggered by dark elves, who want to flood the tunnels. During the resulting chaos they also plan to attack the dwarves, so they can not stop the influx before the water reaches Tvurdu.

		\subsection{Part I -- Surprise!}

		\noindent
		The PCs run down the flooded tunnel. If they proceed with caution, they may try teamwork\TN6 to discover dark elves hiding in one of the wider tunnels behind huge crystals. Otherwise they surprise the dwarves. It is up to the dwarves how they want to overcome the dark elves, but diplomacy will definitely not work. Since the tunnel is flooded ankle-deep, there is a -1 modifier to all physical actions for all involved.

		\keyword{Dark elf:} \HD4, \AD6, \RD6, Scimitar+1. Darksight. Group size: SC/2.

		At least once during combat, each dark elf will try to inflict trauma. They will intimidate the dwarves by crying, \say {You ugly balls of hair! We'll make you drown!}

		\subsection{Part II -- Into the dark}

		After the confrontation, the next goal is either to find the missing buddies (Part III), or stop the water (Part IV) -- depending what the players think of first. Since the network of tunnels is partially flooded and very confusing, the party has to find their way in a \emph{long-term task} (\TN20). Each round represents 15 minutes of searching, listening, digging and running around, while the water level rises. Starting with the second round, all rolls are -1, and from the fourth round on -2. On an automatic failure, the party is confronted with fleeing animals like tunnel rats (\HD2, \AD4, \RD4; Group size: as PCs) or giant centipedes (\HD4, \AD4, \RD4; Group size: SC/2).

		\subsection{Part III -- I can't swim!}

		The party finally hear their buddies cries for help, which lead them to a flooded cave. A few huge crystals rise above the water like small islands, three of which give shelter to dwarven miners who are trapped there.

		If the PCs did not yet solve Part IV there is no time to loose: in about 30 minutes the water will consume the crystal islets and the miners will drown. To rescue them, the PCs might improvise a raft or bridge (\emph{long-term task}\TN15, 10min/round). If the PCs already stopped the water, there is no need to hurry and they can simply roll teamwork\TN8 to rescue the three dwarves.

		Should the PCs not yet got the idea that the water must be stopped, the rescued miners will point that out. However, they are too weak to participate in any more adventures.

		\subsection{Part IV -- That's a lot of water!}

		Following the water's current, the PCs discover one of multiple big holes in the ceiling of one of the tunnels. They were caused by the dark elves. Climbing up there is difficult: the wet and slippery walls call for Agility\TN6 without equipment. Also making sure the lanterns won't go out in all the water might be tricky.

		Going up a level, the party discovers a cavern where dark elves erected a dam, its watergates now wide open. The elves intend to close the gates after all the dwarves are washed away and claim the level below. The dam is guarded by an elite party of dark elves. They will try to stop any courageous dwarf that might make it here.

		\keyword{Dark elf (elite):} \HD6, \AD8, \RD8, Scimitar+1, Armor+1. Dark-sight. Group size: SC.

	\mysection{All's well that ends well?}

		\noindent
		The PCs have to solve two problems to succeed in this scenario: rescue the three dwarves and stop the water for good.
}
