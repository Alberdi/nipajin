% (c) 2009-2016 by Markus Leupold-Löwenthal
% This file is released under CC BY-SA 4.0. Please do not apply one-way compatible licenses.

\renewcommand{\clownsVersion}{v1.0.1}

% CHANGELOG-de
%
% 1.0.1
%   - Lektorat Onno
% 1.0
%   - Erstfassung

% --- language dependent typography stuff --------------------------------------

\renewcommand{\fsNormal}{\fontsize{9pt}{11.25pt plus 0.1pt minus 0pt}}
\renewcommand{\fsSmall}{\fontsize{8.5pt}{9.5pt plus 0.1pt minus 0pt}}

% --- pdf metadata -------------------------------------------------------------

\hypersetup{
	pdfsubject={Ein NIP'AJIN Szenario in einem Zirkus.},
}

% --- fine print ---------------------------------------------------------------

\renewcommand{\clownsCredits}{
	Text:~Markus Leupold-Löwenthal; Lektorat:~Onno Tasler
}

% --- main texts ---------------------------------------------------------------

\renewcommand{\clownsPlayers}{%
	Ein Szenario für zwei bis sechs Clowns.
}
\renewcommand{\clownsHeadline}{Clowns in der Manege}
\renewcommand{\clownsToc}{Szenario: Clowns in der Manege}
\renewcommand{\clownsText}{%
	{\itshape
		Der Zirkus Giocondo tourt seit drei Generationen durch Europa. Den Höhepunkt seiner Anziehungskraft hat er allerdings vor Jahren überschritten. Mit zweitklassigen Artisten und Attraktionen kommt er mehr schlecht als recht über die Runden und kann Mensch und Tier oft gerade noch den Winter bringen.

		Wenn es euch nicht gäbe, hätte der grenzenlos optimistische Direktor Lorenzo Di Benedetto sein rot-weißes, 25 Meter durchmessendes und 10 Meter hohes Zirkuszelt bereits zu sperren können. Als Clowns sorgt ihr dafür, dass dem Publikum die Ausrutscher der Artisten nicht auffallen und ganz nebenbei führt ihr eure eigene, große Nummer während der Show auf.

		Heute Abend ist es wieder so weit. 100 Gäste von jung bis alt haben im Zelt Platz genommen -- auch wenn das nicht mal die Hälfte der Tribüne füllt, so viele waren schon lange nicht mehr da. Der Geruch von Zuckerwatte, Sägespänen und Tierdung liegt in der Luft. Die kleine Kapelle über dem Portal, durch das die Artisten treten werden, spielt, etwas schräg, den \say{Einzug der Gladiatoren}. Das Licht wird abgedunkelt, ein Spot folgt dem Zirkusdirektor, der mit einer packenden Rede das Publikum auf die folgenden Attraktionen neugierig macht~\ldots
	}

	\mysection{Setting}

		\noindent
		Das Szenario spielt während der Zirkus-Aufführung. Die gut ausgeleuchtete Manege durchmisst 10 Meter. Sechs Holzpfeiler stützen das Zeltdach. Der linke und rechte Zuschauerrang wird von einem Gang unterbrochen, der südlich aus dem Zelt zu einem kleinen Vorplatz führt, wo neben der Kassa ein Zuckerwattestand, ein Watschenmann und eine kleine Schießbude stehen. Durch ein großes Portal im Norden des Zeltes, über dem das Orchester spielt, gelangen die Artisten und Tiere in die Manege.

		Es gilt, die Gäste nicht zu vergraulen, obwohl in jeder Nummer etwas schiefgehen wird. Falls die Clowns eine Szene nicht retten, verlässt pro fehlendem Erfolg / verbliebenem \HD-Wert der Gegner ein Zuschauer das Zelt.

	\mysection{Charaktere}

		\noindent
		Die Spieler verkörpern Zirkusclowns und verwenden den W6 als \HD. Ihre besonderen Eigenschaften leiten sich von ihren Clown-Gegenständen ab.

		\tabelle{l X}{
		\thead{Gegenstand} & \thead{Eigenschaft} \\
		}{
			Federschuhe          & Springen+1 \\
			Gummihammer          & Kämpfen+1 \\
			Konfetti-Beutel      & Charisma+1 \\
			Luftballon-Schlangen & Basteln+1 \\
			Mini-Fahrrad         & Schnell+1 \\
			Ukulele              & Musizieren+1 \\
		}

		\noindent
		Als typische Clownsnamen kommen \zB~Binky, Charlie, Giggles, Shorty, Rico oder Wally in Frage.

	\mysection{Ablauf}

		\noindent
		Das Szenario beginnt mit der Ansprache Lorenzos, dann starten die Show-Blöcke. Entweder bestimmt der Spielleiter die Reihenfolge oder er würfelt sie mit einem W6 aus.

		\subsection{Szene I: Der Magier}

		Der dunkel gekleidete Magier \keyword{Monsieur Mystérieux} betritt, begleitet von getragener Musik, die Manege. Er zieht seiner bezaubernde Assistentin, aber auch dem Publikum Tücher aus den Ohren, verkettet scheinbar lose Ringe und lässt Wasser in Zeitungstüten verschwinden. Beim Höhepunkt der Nummer, bei dem ein Hase aus dem Zylinder gezogen werden soll, wirkt der Magier panisch -- der Hase  hoppelt im Schatten des Portals aus dem Zelt.

		Den Hasen einzufangen und unauffällig zu Mystérieux zu bringen ist eine langfristige Aktion (\TN15, 30 Sek./Runde, 3 Runden), sonst muss der Magier unter Buh-Rufen die Manege verlassen.

		\subsection{Szene II: Die Elefanten}

		Der zierliche Dompteur \keyword{Albert van der Velde}, in eine rote Jacke mit goldenen Knöpfen gekleidet, führt drei Elefanten in die Manege. Sie geben, wie abgerichtete Hunde, allerlei Kunststücke zum Besten, etwa Händchen geben oder aufeinander zu klettern. Als einer der Elefanten auf einem Ball balanciert, bemerken die Clowns, dass von Besuchern achtlos fallengelassene Essensreste unter der Tribüne Mäuse angelockt haben.

		Um zu verhindern, dass die Elefanten beim Anblick der Mäuse in Panik geraten, müssen die Clowns rasch unter die Tribüne steigen und die Nager in 3 Runden überwinden. \nsc{Maus}{1}{2}{4}, Anzahl: Clowns×3.


		\subsection{Szene III: Der Drahtseilakt}

		Sieben Meter über dem Boden wirbeln bei hektischer Musik die \keyword{Fabulosen Dimitris} durch die Luft und fangen einander auf, während sie an zwischen den Zeltpfeilern angebrachten Schaukeln schwingen. Lorenzo kündigt mit großen Worten das Finale an, bei dem drei Artisten nach einem gewagten Sprung als menschliche Pyramide auf einem Drahtseil landen. Die Clowns bemerken derweilen Schrauben und Muttern am Fuße dreier Pfeiler liegen, die wohl jemand vergessen hat, oben anzubringen.

		Damit die Artisten nicht in ihr Unglück stürzen, müssen die Clowns die Schrauben mit einer langfristigen Aktion fixieren, während Lorenzos seine Rede schwingt (\TN15, 30 Sek./Runde, 3 Runden).

		\subsection{Szene IV: Die Löwen}

		Dompteur \keyword{Albert van der Velde} betritt (erneut) die Manege. Ohne Sicherheitsnetz führt er mit einem Löwenpärchen Kunststücke auf, lässt sie brüllen und durch brennende Reifen springen. Da hüpfen Jugendliche von der Tribüne, um ein Selfie mit den Miezekätzchen zu machen.

		Falls die Clowns die Jugendlichen nicht unauffällig aus der Arena entfernen, werden die Löwen sie ernsthaft verletzen. Allerdings haben die Jugendlichen keinen Respekt vor den Clowns und werden sich mit Schlägen und Tritten gegen Gewalt (und auch viele Scherze) wehren. \nsc{Jugendlicher}{3}{4}{6}, Anzahl: wie Clowns.

		\subsection{Szene V: Der Feuerspucker}

		Der stämmige, bärtige und am Oberkörper unbekleidete, liebevoll \keyword{Pa‘ Pyrus} genannte Feuerspucker, betritt die Manege. Während der Vorstellung kommt er ins Stocken, räuspert sich und hustet dann. Dabei spuckt er einen Zeltpfeiler in Brand.

		Falls die Clowns die Flammen nicht innerhalb von zwei Minuten ersticken, muss das Zelt zumindest kurzzeitig evakuiert werden (\TN21, 30 Sek./Runde, 4 Runden).

		\subsection{Szene VI: Die Clowns}

		Die Spieler haben hier die Gelegenheit, sich ihre eigene Nummer auszudenken und zu inszenieren. Für fünf Minuten genießen die volle Aufmerksamkeit des Publikums, in denen sie zur Abwechslung mal nicht blenden und täuschen müssen (langfristige Aktion\TN25, 1 Min./Runde, 5 Runden).

	\mysection{Ende gut, alles gut?}

		\noindent
		Überwundene Clowns werden bewusstlos aus der Manege getragen. Das Szenario endet erfolgreich, wenn die Show nicht vorzeitig gestoppt wird. Ob die Spieler mit der verbleibenden Zuschauerzahl zufrieden sind, bleibt ihnen überlassen.

}
