% (c) 2009-2015 by Markus Leupold-Löwenthal
% This file is released under CC BY-SA 4.0. Please do not apply one-way compatible licenses.

\renewcommand{\hanamiVersion}{v1.0}

% CHANGELOG-de
%
% 1.0
%   - Erstfassung

% --- language dependent typography stuff --------------------------------------

\renewcommand{\fsNormal}{\fontsize{9pt}{11.25pt plus 0.1pt minus 0pt}}
\renewcommand{\fsSmall}{\fontsize{8.5pt}{9.5pt plus 0.1pt minus 0pt}}

% --- pdf metadata -------------------------------------------------------------

\hypersetup{
	pdfsubject={Ein NIP'AJIN Szenario im feudalen Japan.},
}

% --- fine print ---------------------------------------------------------------

\renewcommand{\hanamiCredits}{Text: Nicole Goci}

% --- main texts ---------------------------------------------------------------

\renewcommand{\hanamiPlayers}{%
	Ein Szenario für 3-6 Charaktere.
}
\renewcommand{\hanamiHeadline}{Hanami}
\renewcommand{\hanamiToc}{Szenario: Hanami}
\renewcommand{\hanamiText}{%
	{\itshape
		Eine mittelalterliche japanische Welt, regiert von einer Kaiserin und ihren Samurai zu einer Friedenszeit. Kaiserin Hanako ist das Sprachrohr der Götter auf Erden und hat viele geheimnisvolle Fähigkeiten. Neben verschneiten Bergen und geheimnisvollen Wäldern gibt es auch idyllische Graslandschaften und malerische Küstenregionen. Zwischen drinnen ragen Pagodengebäude, Torii, Schlösser, Festungen, Dörfer und Städte auf.

		Ihr nähert euch gerade der Hauptstadt im Zentrum des Festlandes, während der Frühling um euch herum seine volle Pracht zeigt. Der Grund für eure Reise ist die Einladung der Kaiserin zum Kirschblütenfest. Doch schon während der Reise ist euch aufgefallen das kein Kirschbaum im ganzen Reich blüht.

		Übermorgen soll das Fest stattfinden und die Samurai der Stadt sind mit den Vorbereitungen beschäftigt für das Fest schwer beschäftigt. Der Tempelvorsteher wendet sich jedoch an euch und bittet euch um Hilfe:

		\say{Werte Samurai, wie ihr wisst wird übermorgen das Kirschblütenfest stattfinden und die Kirschbäume blühen nicht. Die Gärtner der Stadt haben bereits alles versucht und bestätigt dass die Bäume gesund sind. Eine natürliche Ursache ist demnach auszuschließen. Ich bitte euch herauszufinden wo das Problem liegt und dieses zu beseitigen. Ihr sollt dafür natürlich entlohnt werden}, mit diesen Worten verabschiedet sich der Tempelvorsteher dann auch wieder.
	}

	\mysection{Charaktere}

		\noindent
		Jeder spielt einen Samurai, also einen Mann oder eine Frau aus der Adelskaste. Als ehrenhafte Adelige die an den Kodex des Bushido gebunden sind fällt es euch schwer unehrenhaftes Verhalten anzuwenden (Schleichen, Hinterhalt und ähnliches -4).

		\tabelle{l c X}{
		\thead{SC} & \thead{\HD} & \thead{Beschreibung} \\
		}{
			Diplomat      &  4 & Soziales+2, Kraft-1, Kunst+1 \\
			Schwertweiser & 10 & Katana+2, Kunst-1, Kraft+1 \\
			Bogenmeister  &  8 & Bogen+2, Kampf-1, Kunst+1 \\
			Onmyouji      &  6 & Zwei Effekte zweier Kategorien* \\
			Priester      &  6 & Effekt: Heilung, Wissen+1 \\
			Gelehrter     &  4 & Wissen+2, Kraft-1, Kunst+1 \\
		}

		\noindent
		Alle Zauber werden auf Papier kalligraphiert (Ofuda), Schutzzauber können daher an andere weitergereicht werden. Magier und Priester können alle Effekte fünf mal anwenden.

	\mysection{Setting}

		\noindent
		Das Szenario spielt in der Hauptstadt des Kaiserreiches. Im Zentrum der Stadt thront der imposante Palast der Kaiserin auf einem Hügel, den Palast umgibt der Regierungsbezirk und der Tempelbezirk. Dahinter liegen dann der Handels- und Handwerksbezirk, der Kultur- und Vergnügungsbezirk und die Wohnbezirke. Die Hauptstraße der Stadt ist gesäumt von Kirschbäumen, dazwischen werden Marktstände für das Fest aufgebaut und auch Bereiche für Künstler und Wettkämpfe abgesperrt. Es summt vor Leben, denn alle Samurai wurden in die Hauptstadt eingeladen und somit ist die Zahl der Menschen auf das doppelte angestiegen.

	\mysection{Ablauf}

		\subsection{Teil 1}

		Die Gruppe muss nun Informationen sammeln, wenn sie in der Stadt die Bewohner fragen werden sie in Erfahrung bringen, dass Abends gelegentlich weinen in der Nähe der Kirschbäume gehört, jedoch niemand gesehen wurde.

		Wenn die SCs über diese Information diskutieren bzw nachdenken können sie mit einem Wurf auf Wissen\TN4 erfahren das wohlmöglich ein niederer Gott oder ein Geist die Ursache sein könnte, wer ein Ergebnis von 6 erreicht weiß dass die niedere Gottheit der Kirschblüten damit zu tun haben könnte.

		Die Spieler können nun Nachforschungen in der Tempelbibliothek anstellen und werden mit vereinten Kräften feststellen das die niedere Gottheit der Kirschblüten weder Schrein noch Namen hat.

		\subsection{Teil 2}

		Sobald die Gruppe abends die Tempelbibliothek verlässt um sich für den Abend zurückzuziehen, hören sie in der Nähe des ersten Kirschbaums ein leises Weinen. Die SCs haben nun die Möglichkeit nach der Quelle des Weinens Ausschau zu halten  mit einem Wurf von 5 finden sie auf einem der Äste des nächsten Kirschbaumes ein kleines Mädchen im kirschblüten-rosa-farbenen Kimono mit kurzem schwarzen Haar. Das Mädchen weint und schluchzt. Mit einem Wurf von 5 erkennen die SCs, dass das Mädchen die Gottheit der Kirschblüten ist.

		Im Gespräch mit dem Mädchen werden sie herausfinden das ein Mann in schwarzer Kleidung mit dem Symbol des Totenkopfes darauf der Gottheit versprach ihr einen Namen zu geben, einen Tempel zu bauen und sie zu einer Dunklen Göttin zu machen, wenn sie dafür verhindert dass die Kirschen zum Fest blühen. Die Gottheit wünscht sich einen Namen und einen Schrein, denn nur einmal im Jahr finden Feste statt und auch da wird nicht sie verehrt, sondern die Feiernden trinken und essen.

		Wenn die SCs der Gottheit versprechen ihr einen Namen zu geben beschreibt das Mädchen ihnen wo sie den Mann immer trifft und verschwindet danach wieder.

		\subsection{Teil 3}

		Die SCs wissen das die Kaiserin die Macht hat Menschen nach ihrem Tot zu niederen Gottheiten zu erheben und sollten daher eine Audienz bei Kaiserin Hanako erbitten. Hierzu sprechen sie am Besten mit dem Tempelvorsteher.

		Wenn die Spieler dem Tempelvorsteher die Sachlage erklären, werden sie am Abend vor dem Kirschblütenfest eine Audienz bei der Kaiserin bekommen. Am Weg aus dem Temeplbezirk heraus finden sie einen Mann auf den die Beschreibung der Gottheit passt. (\HD8, \AD6, \RD6)

		Der Mann wird sich selbst mithilfe von Gift töten, sollten die SCs versuchen ihn lebend zu fangen.

		Kaiserin Hanako wird bei der Audienz die SCs um Vorschläge für Namen bitten, \zB\ Sakura (Kirsche/ Kirschblüte), Hanami (Blütenschau), Harumi (Frühlingsschöhnheit), Haruki (Frühlingsbaum), Haruko (Frühlingskind). Die Kaiserin bittet die Gruppe vor ihrer eigenen Ansprache die Geschichte der Gottheit zu erzählen und ihren Namen zu verkünden.

		\subsection{Teil 4}

		Mithilfe von Kunst und Soziales können die SCs eine angemessene Präsentation am Fest halten nach ihrem letzten Wort wird die Göttin vor der Menge in der Luft als junge Frau erscheinen und Kirschblüten über die Menschen verstreuen.

	\mysection{Ende gut, alles gut?}

		\noindent
		Die Kaiserin verkündet einen Erben empfangen zu haben und dass das Kaiserreich in eine friedliche Zukunft blickt.
}
