% (c) 2009-2016 by Markus Leupold-Löwenthal
% This file is released under CC BY-SA 4.0. Please do not apply one-way compatible licenses.
% Original version by André Frenzer.

\renewcommand{\wildwestVersion}{v1.1.1}

% CHANGELOG-de
%
% v1.1.1
%   - kleines Lektorat
% v1.1
%   - Beispielcharaktere
%   - 3-4 Spieler
%   - Indianer-Anführer
%   - kleinere Korrekturen
% v1.0.1
%   - Lektorat Onno
% v1.0.1
%   - Lektorat Markus
% v1.0
%   - Erstfassung von André Frenzer

% --- language dependent typography stuff --------------------------------------

\renewcommand{\fsNormal}{\fontsize{9pt}{11.25pt plus 0.1pt minus 0pt}}
\renewcommand{\fsSmall}{\fontsize{8.5pt}{9.5pt plus 0.1pt minus 0pt}}

% --- pdf metadata -------------------------------------------------------------

\hypersetup{
	pdfsubject={Ein NIP'AJIN Szenario im Wilden Westen.},
}

% --- fine print ---------------------------------------------------------------

\renewcommand{\wildwestCredits}{
	Text:~André Frenzer; Überarbeitung:~Markus Leupold-Löwenthal; Lektorat:~Onno Tasler%
}

% --- main texts ---------------------------------------------------------------

\renewcommand{\wildwestPlayers}{%
	Ein Szenario von André \say{Seanchui} Frenzer für 3 bis 4 unerschrockene Cowboys.
}
\renewcommand{\wildwestHeadline}{Der etwas andere Banküberfall}
\renewcommand{\wildwestToc}{Szenario: Der etwas andere Banküberfall}
\renewcommand{\wildwestText}{%
	{\itshape
		Wir schreiben das Jahr 1877. Der Wilde Westen ist noch wunderbar wild und es ist eine dieser typischen, von Gott und den meisten Menschen verlassenen Siedlungen irgendwo zwischen Mexiko und Texas, in die euch der Lockruf des Goldes verschlagen hat. Euer kleiner Claim erwies sich als lukrativ und ihr habt schon ein hübsches Sümmchen Gold in der kleinen Bank im Ort hinterlegt. Heute sucht ihr ein letztes Mal die Stadt auf, bevor ihr in den nächsten Tagen eure Zelte abbrecht.

		Eure Aufmerksamkeit erregt einmal mehr das im heißen Wüstenwind müde klappernde, verwitterte Holzschild mit der Aufschrift \say{Saloon}. Ein paar verhältnismäßig kühle Getränke später ist die Mittagshitze fast vergessen und ihr habt Gelegenheit, aus den stumpfen Fenstern des Gastraumes die Stadt zu beobachten. Die Hitze lässt die Luft flimmern, die Straßen sind staubig und kaum jemand ist zu sehen.

		Dann bricht mit einem Mal die Hölle los! Rothäute, Indianer, egal wo ihr hinseht. In voller Kriegsbemalung, mit lautem Geheul und mit ihren Gewehren in die Luft schießend jagt eine euch zahlenmäßig weit überlegene Horde durch die Straßen der Stadt. Ihr Ziel: die Bank, in der euer sauer erarbeitetes Gold liegt~\ldots
	}

	\mysection{Charaktere}

		\noindent
		In diesem Szenario verkörpern die Spieler befreundete Goldgräber und Cowboys. Als Ausrüstung besitzen sie je ein Pferd, ein Lasso, einen Revolver und ausreichend Munition (2 Wunden). Außerdem führen sie das Nötigste für einen kurzen Ritt durch die Wüste bei sich: Hüte, Tücher und einen Wasserschlauch.

		\tabelle{l c X}{
		\thead{SC} & \thead{\HD} & \thead{Beschreibung} \\
		}{
			Jack &  8 & Champion im Hufeisenwerfen \\
			Jake &  8 & extrovertierter Lassoschwinger \\
			Jay  &  8 & trinkfester Raufbold \\
			Joe  &  8 & stummer Rodeo-Reiter \\
		}

		\noindent
		Die Charaktere erhalten keine besonderen Vor- oder Nachteile, aber Aktionen, die zu ihrer Beschreibung passen, sind um +1 erleichtert.

	\mysection{Setting}

		\noindent
		Das Szenario spielt in und um einer kurzlebigen Goldgräberstadt. Die Häuser sind großteils eilig zusammengenagelte Bretterbuden mit Flachdächern. Lediglich die Bank, ein Ziegelbau, erweckt einen etwas solideren Eindruck.

	\mysection{Ablauf}

		\noindent
		Gleich zwei Parteien haben es auf die Bank und damit auf das Ersparte der Charaktere abgesehen: Eine Bande räuberischer Indianer sowie der verrückte Wissenschaftler Ian McDermid, der mithilfe einer dampfmechanischen Ameise das Gebäude zu knacken gedenkt.

		Die Charaktere werden es zunächst mit den Indianern zu tun bekommen. Diese wissen nichts vom Wissenschaftler, sein Auftauchen überrascht sie ebenso wie die Charaktere. Sobald das dampfmechanische Ungetüm in den Banküberfall eingreift, wird der Handlungsverlauf sehr offen. Es ist denkbar, dass sich die vier Cowboys mit den Indianern verbünden, um gegen den Dampfroboter vorzugehen. Genauso gut können sich die Charaktere aber auch dafür entscheiden, zunächst das Indianerproblem zu lösen, um sich anschließend dem verrückten Wissenschaftler in den Weg zu stellen.

		Das Szenario startet im Saloon. Drei Runden benötigen die Charaktere, um von dort zur, ein paar Häuserblöcke entfernten Bank zu gelangen.

		\subsection{Die Indianer}

		Eine Gruppe Indianer hat ihre Pferde vor der Stadt geparkt und greift in einem wilden Sturmangriff an. Ihre Anführer besitzen Gewehre und sind bereits ins Bankgebäude vorgedrungen. Der Rest macht die engen, staubigen Straßen unsicher und trägt nur Messer bei sich. Es gilt, die Indianer zu überrumpeln, damit ihre zahlenmäßige Überlegenheit nicht zum Tragen kommt.

		Es wird in Runden von einer Minute vorgegangen. Jeder Charakter kann sich \zB\ an einen Indianer heranschleichen (\TN3), um diesen ohne Gegenwehr anzugreifen. Alternativ können Goldschürfer versuchen, einen Hinterhalt zu legen (langfristige Aktion, \TN12), um W4+1 Indianer in eine Falle laufen zu lassen und auszuschalten. Wer versucht, sich hinter Fässern zu verschanzen (\TN3, +1 auf Verteidigung) oder direkt zur Bank läuft (2 Runden), dem stellen sich drei Indianer gleichzeitig in den Weg.

		\nsc{Indianer}{1}{6}{4}, Messer, ergibt sich in aussichtsloser Lage, Anzahl:~SC×4

		\nsc{Indianer-Anführer}{4}{6}{6}, Gewehr (2 Schaden), Schießen vom Pferd -1, Anzahl:~wie SC

		\subsection{Der verrückte Wissenschaftler}

		Ian McDermid ist jedes Mittel recht, um an Geld für seine Forschungen zu gelangen. Er steuert ein dampfgetriebenes, mechanisches Ungetüm, das an eine Dampflokomotive auf sechs insektoiden Beinen erinnert. Er kann ab Runde 3 bemerkt werden (\TN4). In Runde sechs erreicht er die Bank und wird mit seiner Konstruktion durch die Fassade preschen. In Runde acht verlädt er mit einem Greifarm den Tresor in die Lokomotive und wird anschließend die Flucht in sein Geheimversteck antreten.

		Weder die Charaktere noch die Indianer können dem Ungetüm direkt gefährlich werden. Sie werden mit ihren Pferden die Verfolgung aufnehmen müssen (langfristige Aktion, \TN12). Einmal eingeholt können die Charaktere auf die Maschine springen oder klettern (\TN6 oder 1 Wunde). Anschließend müssen sie sich dem Greifarm der Erfindung erwehren, während sie sich Zugang in das Innere verschaffen, was mit improvisiertem Werkzeug schwerfällt (-1) und vereinte Kräfte erfordert (langfristige Aktion, \TN20, 1 Minute/Runde). Sind keine Indianer anwesend, die das Ungetüm ablenken, wird McDermid versuchen, die Charaktere abzuwerfen: Auf der schwankenden Konstruktion stehen erschwert alle Proben zusätzlich um -1.

		\nsc{Dampfameise}{12}{12}{10}, Greifarm: 2 Schaden, immun gegen normale Waffen

		\nsc{Ian McDermid}{4}{6}{4}

	\mysection{Ende gut, alles gut?}

		\noindent
		Haben die Charaktere sich Zutritt zu dem dampfmechanischen Ungetüm verschafft, ist das Überwinden Ian McDermids reine Formsache. Die vier Goldgräber halten ihr Gold endlich wieder in Händen und können sich auf ein angenehmeres Leben freuen.
}
