% (c) 2009-2015 by Markus Leupold-Löwenthal
% This file is released under CC BY-SA 4.0. Please do not apply one-way compatible licenses.

\renewcommand{\caribsVersion}{v1.0-wip}

% CHANGELOG-de
%
% 1.0
%   - Erstfassung

% --- language dependent typography stuff --------------------------------------

\renewcommand{\fsNormal}{\fontsize{9pt}{11.25pt plus 0.1pt minus 0pt}}
\renewcommand{\fsSmall}{\fontsize{8.5pt}{9.5pt plus 0.1pt minus 0pt}}

% --- pdf metadata -------------------------------------------------------------

\hypersetup{
	pdfsubject={Ein pulpiges NIP'AJIN Szenario auf einer einsamen Karibikinsel.},
}

% --- fine print ---------------------------------------------------------------

\renewcommand{\caribsCredits}{nobody}

% --- main texts ---------------------------------------------------------------

\renewcommand{\caribsPlayers}{%
	Ein Szenario für drei bis fünf Entdecker.
}
\renewcommand{\caribsHeadline}{Der Tiegel der Kariben}
\renewcommand{\caribsToc}{Szenario: Der Tiegel der Kariben}
\renewcommand{\caribsText}{%
	{\itshape
		Als Mitglieder der Trailblazer Society seid ihr vom Britischen Museum in London angeheuert worden. Ihr sollt für die Karibische Sammlung ein attraktives Exponat besorgen: Der Kurator hat in alten Piratentagebüchern Hinweise auf eine noch unbekannte Insel gefunden, auf der sich der legendäre Tiegel der Kariben befinden soll, der mal golden, mal juwelenbesetzt beschrieben wird.

		Von Puerto Rico aus seid ihr mit eurem Wasserflugzeug einer verwaschenen Karte gefolgt und habt die Insel tatsächlich entdeckt: malerische, goldgelbe Sandstrände umgeben einen Urwald, aus dem ein niedriger Vulkankegel ragt. Zu eurer Überraschung ist die Insel bewohnt, und die Einwohner nach dem ersten Schreck über euer Flugzeug freundlich und aufgeschlossen. Nachdem ihr Gastgeschenke ausgetauscht, mit Händen und Füßen erklärt habt, wegen dem Tiegel hier zu sein, und gemeinsam eine milchige Flüssigkeit getrunken habt, beginnen die Einwohner zu tanzen und sich dabei über ihre nackten Bäuche zu streichen. Dann wird~\ldots~euch schwarz vor Augen.

		Als ihr wieder zu euch kommt, befindet ihr euch an den Händen gefesselt im Inneren eines gigantischen Tontopfes, der am Fuße des Vulkans von einem Lava-Strom beheizt wird. Von Gold und Edelsteinen keine Rede. Während sich um euch die würzige Brühe langsam erwärmt, ist einer der Kariben auf einem Steg über euch damit beschäftigt, Wurzeln und Gemüse klein zu schneiden und auf euch herabrieseln zu lassen. Ein zweiter sorgt mit einem langen Schöpflöffel dafür, dass eure Köpfe stets mit Brühe benetzt sind. Jenseits des Urwaldes und des Dorfes der Kariben, die ihr beide von hier aus gut überblicken könnt, treibt euer Fluchtflugzeug mit der Ebbe langsam vom Strand weg, dem Sonnenuntergang entgegen~\ldots
	}

	\mysection{Charaktere}

		\noindent
		Die Spieler verkörpern pulpige Abenteurer aller Art mit unterschiedlichen Spezialgebieten. Zu Beginn des Szenarios sind sie unbekleidet, haben keinerlei Ausrüstung bei sich müssen diese improvisieren (-1 auf Aktionen).

		\tabelle{l c X}{
		\thead{SC} & \thead{\HD} & \thead{Vorteile} \\
		}{
			Blinder Passagier &  4 & Heimlichkeit+2, Körperliches-1 \\
			Großwildjäger     &  8 & Schießen+2, Soziales-1 \\
			Linguist          &  6 & Sprachen+1, Soziales+1, Körperliches-1 \\
			Pilot             &  4 & Fliegen+1, Technik+1, Schwimmen-1 \\
			Professor         &  8 & Wissen+1 \\
		}

		\noindent
		Der blinde Passagier ist \zB~ein Kind aus Puerto Rico und hatte sich aufs Flugzeug geschlichen.

	\mysection{Setting}

		\noindent
		Das Szenario spielt auf einer, von Kannibalen bevölkerten, bildhübschen Karibikinsel. Die Kariben tragen Röcke, Arm- und Fußreifen aus Palmenblättern und sprechen eine unbekannte Sprache. \nsc{Kannibale}{2}{6}{4}.

		Das Dorf der Kariben besteht aus zwei Dutzend Häusern, die aus dünnen Baumstämmen oder Ästen gefertigt sind und mit Blättern gedeckt sind.

	\mysection{Ablauf}

		\noindent
		Dieses Szenario handelt von der Flucht einer Gruppe Entdecker vor den wilden Einwohner einer Insel. In Szene I gilt es, aus dem Kochtopf zu entkommen. Szene II und III können in beliebiger Reihenfolge gespielt werden, letztere auch mehrfach. In der Schlussszene IV gelingt hoffentlich die Flucht.

		Der Sonnenuntergang kann den Charakteren als Orientierung dienen und taucht alles in goldgelbes Licht. Sollten sich die Charaktere zu lange Zeit lassen, wird es erst dämmrig, dann Nacht (-1/-2 auf alle Aktionen).

		\subsection{Szene I: Der Tiegel}

		Der Topf und die zwei Kariben müssen in einem Konflikt überwunden werden. Den Topf zu zerbrechen oder durch Wippen umzuwerfen ist eine langfristige Aktion (\TN20). Ab der zweiten Runde versuchen die Kannibalen, mit Schöpfer und Rührstangen, ihr Essen bewusstlos zu schlagen. Alternativ können die Charaktere versuchen, den Koch durch Erschrecken oder Handgreiflichkeiten zu sich in den Topf zu bringen, um an sein Messer zu gelangen. Solange Charaktere gefesselt sind, haben sie -1 auf physische Aktionen.

		Von der Kochstelle führt ein Weg durch den Urwald zum Dorf. Von dort ist in der Ferne Gesang zu hören, der durch einen Schuss beendet wird.

		\subsection{Szene II: Das Dorf}

		Einer der Gründe, ins Dorf einzudringen, wäre die begründete Vermutung, dass sich dort die Ausrüstung der Charaktere befindet. Tatsächlich wird diese gerade von einer handvoll Kariben inspiziert. Einer davon hatte unabsichtlich eine Pistole abgefeuert und sich ins Bein geschossen. Er wird derzeit vom Medizinmann untersucht und hofft, nicht in den Kochtopf zu wandern.

		Vorausgesetzt die Charakter werden irgendwie mit den Kannibalen fertig (Anzahl: SC×2), können sie hier ihre Habe wieder einsammeln, sich standesgemäß anziehen und haben keinen Malus wg. behelfsmäßiger Ausrüstung mehr. Zudem kann sich jeder Charakter einen passenden Gegenstand wünschen, der +1 auf bestimmte Proben gibt, oder eine Waffe, die 2 Wunden pro Treffer verursacht.

		\subsection{Szene III: Im Urwald}

		Der Urwald hat pro Stunde eine zufällige Überraschung parat, wenn Charaktere sich hier verstecken oder ausrüsten.

		\tabelle{c X}{
		\thead{W4} & \thead{Hindernis} \\
		}{
			1 & SC/2 Jäger mit giftigen Blasrohren (lähmen für eine Stunde: -2 auf alle Würfe) \\
			2 & \nsc{Boa constrictor}{4}{10}{8} \\
			3 & Kleintierfallen (Agilität\TN5 oder 1 Wunde)  \\
			4 & \nsc{Süßwasserkrokodil}{6}{8}{6} \\
		}

		\subsection{Szene V: Am Strand }

		Das Flugzeug ist bereits 500 Meter aufs Meer getrieben. 10 schmale Boote (für je max. 3 Personen) sind am Strand unter Blättern verborgen (Aufmerksamkeit\TN5).

		Es wird in Runden von je ca. 1 Minute vorgegangen. Die 500 Meter zu schwimmen dauert zehn Runden (\TN3), rudern per Boot nur fünf (\TN4). Ein Boot ans Wasser zu zerren kostet eine Runde (\TN5), ebenso eines leck zu schlagen (\TN4).

		In Runde drei kündigen sich schreiende Kariben an, in Runde vier kommen sie auf den Strand gelaufen. Ein Teil beginnt, jede Runde auf die Fliehenden zu Schießen (1 Pfeil pro Charakter), solange sie noch in Reichweite -- 100m -- sind. Auf vier Booten, zur Not schwimmend, folgen die Kannibalen. Pro Boot rudern zwei, während ein dritter versucht, zu schießen (-1 wg. Boot).

		Beim Flugzeug angekommen haben Charaktere Zugriff auf ein Erste-Hilfe-Set. Leider stellen sie auch fest, dass eine Flügelklappe beschädigt ist, klemmt und repariert werden muss (langfristige Aktion \TN12), ehe sie starten können.

	\mysection{Ende gut, alles gut?}

		\noindent
		Das Szenario endet erfolgreich, wenn die Charaktere mehr oder wenig unbeschadet die Insel mit ihrem Flugzeug verlassen -- vorzugsweise Richtung Sonnenuntergang.
}
