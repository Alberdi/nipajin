% (c) 2009-2016 by Markus Leupold-Löwenthal
% This file is released under CC BY-SA 4.0. Please do not apply one-way compatible licenses.

\renewcommand{\androidsVersion}{v1.1}

% CHANGELOG-de
%
% 1.1.1
%   - Lektorat Onno und Ask
% 1.1
%   - Überarbeitung Markus
% 1.0
%   - Erstfassung von Askin

% --- language dependent typography stuff --------------------------------------

\renewcommand{\fsNormal}{\fontsize{9pt}{11.0pt plus 0.1pt minus 0pt}}
\renewcommand{\fsSmall}{\fontsize{8.5pt}{9.5pt plus 0.1pt minus 0pt}}

% --- pdf metadata -------------------------------------------------------------

\hypersetup{
	pdfsubject={Ein NIP'AJIN Szenario für solarbetriebene Androiden.},
}

% --- fine print ---------------------------------------------------------------

\renewcommand{\androidsCredits}{
	Text:~Aşkın-H. Doğan; Überarbeitung:~Markus Leupold-Löwenthal%
}

% --- main texts ---------------------------------------------------------------

\renewcommand{\androidsPlayers}{%
	Ein Szenario für zwei bis vier Androiden von Aşkın-H. Doğan.
}
\renewcommand{\androidsHeadline}{Sonnenfinsternis auf Rimol}
\renewcommand{\androidsToc}{Sonnenfinsternis auf Rimol}
\renewcommand{\androidsText}{%
	{\itshape
		Dank extraterrestrischer Sojalegierungen machte die Androidistik einen Quantensprung und ebnete den Weg für künstliche Lebensformen, die im Namen der Wissenschaft öde Planeten erforschen.

		Ihr seid Androiden der berühmten, vollautomatisierten Wissenschaftseinheit Science-Tastic, die gänzlich mit Solarenergie funktionieren. Seit ein paar Jahren forscht ihr auf dem menschenleeren Zwergplaneten Rimol, der von zwei Sonnen permanent erleuchtet wird und nur eine dünne Atmosphäre besitzt. Von sauerstoffarmer Tomatenzucht über Clusterverkalkung bis zum Sozialverhalten ausgestorbener Tiere lässt euer Forschungsdrang nichts aus. Eure Ergebnisse sind bisher \ldots durchwachsen.

		Auf die lang erwartete, achtstündige Sonnenfinsternis habt ihr euch sehr gefreut. Die Monde schieben sich vor die Zwillingssonnen und tauchen alles in malerische Dunkelheit. Im schwachen Licht eures Notaggregats schaltet ihr in den Stromsparmodus und betrachtet gemeinsam mit eurem Putzroboter Staubi das seltene Spektakel.

		Ihr ahnt nichts Schlimmes, als plötzlich~\ldots
	}

	\mysection{Charaktere}

		\noindent
		Die Spieler verkörpern solarbetriebene \keyword{Androiden}. Sie befinden sich im Stromsparmodus und müssen auf eine wichtige Funktion verzichten: das Sprachmodul. Sie können nur Ein-Wort-Sätze oder frühkindliches Geplärre wie \say{Wähhhh!} von sich geben.

		\tabelle{l c X}{
		\thead{SC} & \thead{\HD} & \thead{Beschreibung} \\
		}{
			Twenk VnK    &  6 & Flink+1, Technik-1 \\
			Deep Psy     &  6 & Kommunikation+1, Kämpfen-1 \\
			PoWW         &  8 & Kraft+1, Kommunikation-1 \\
			Na-A-La 2.0  &  4 & Reparatur+1, Kraft-1 \\
		}

	\mysection{Setting}

		\noindent
		Das Szenario spielt in einer voll automatisierten \keyword{Forschungsstation} auf Rimol. Auch diese ist eigentlich auf Solarenergie angewiesen und wird derzeit mit Notstrom versorgt. Dank diesem gibt es in ihrem Inneren genug Licht für die Solarzellen der Androiden. Draußen ist es jedoch dunkel, sie erleiden dort −1 auf alle Würfe. Es gibt keine nennenswerten Druckunterschiede. Eine Karte der Anlage ist nicht nötig: es wird davon ausgegangen, dass die Androiden stets rechtzeitig zum Ort einer Szene laufen können.

		Neben den Androiden bevölkern naturverbundene \keyword{Aliens} den Planeten. Diese lichtscheuen Kreaturen sind kulturschaffende Wesen, die äußerlich an metergroße Säbelzahnfrösche erinnern. Werden sie überwunden, teleportieren sie sich mit schamanistischen Kräften aus dem Spiel und hinterlassen dabei ein grünbläuliches Flimmern. Sie wollen die Androiden im Schutz der Dunkelheit von ihrem Planeten vertreiben. \nsc{Alien}{4}{6}{6}, Speer+1, Lichtscheu-1

	\mysection{Ablauf}

		\noindent
		Mit dem Eintritt in die Sonnenfinsternis beginnt das Szenario. Die Reihenfolge der ersten vier Szenen kann der SL selbst bestimmen, mit einem W4 auswürfeln oder einzelne ausfallen lassen. Szene V ist das Finale und wird auf jeden Fall gespielt.

		Zwischen den Szenen dürfen sich die Androiden gegenseitig reparieren (\TN4, 1 Versuch pro zu reparierenden Androide pro Szene, Erfolg: +1 \HD).

		\subsection{Szene I: Tintenfische}

		\noindent
		Während die Androiden im Aufenthaltsraum rasten, schlagen die Aliens ein kleines Fenster ein und werfen  faustgroße, quietschbunt gepunktete, fliegende Tintenfische in die Station. Die sämige rosa Tinte, die sie verspritzen, verklebt Solarzellen und Gelenke der Androiden in Form von Trauma. Die Tinte erschwert alle Aktionen (-1 pro Traumapunkt), kann aber mit einer erfolgreichen Aktion komplett weggeputzt werden (Reinigen\TN5). Werden alle SC durch Tinte überwunden, müssen sie tatenlos zusehen, wie die Tintenfische die schöne Station versauen und durch das Fenster fliehen, ehe die Selbstreinigungssprinkeranlage angeht, die Wände sowie Androiden reinigt.

		\nsc{Tintenfisch}{1}{2}{3}, Tinte+1, Anzahl: SC×4

		\subsection{Szene II: Staubi}

		\noindent
		Das selbst gebaute Haustier Staubi ist eine Mischung aus Wachhund und Staubsauger. Es muss im Inneren der Station gehalten werden, da der Planet zu schmutzig ist. Staubis Ultraschall-Gebell dringt durch die Wände, als ihn Aliens aus der Station teleportieren. Umgeben vom Staub und Dreck der Außenwelt tobt er und wird sich zu Tode saugen, wenn die SC ihn nicht wieder einfangen! Langfristige Aktion: \TN22, 1 Minute / Runde, 1 Wunde pro Fehlschlag für alle Teilnehmenden.

		\nsc{Staubi}{4}{6}{6}

		\subsection{Szene III: Schnecken}

		\noindent
		Die Solarkäfige der joruanischen Nacktschnecken sind ausgefallen. Die Tiere machten sich über die Tomaten im Gewächshaus her. Durch den Verzehr der blauen Delikatessen mutieren sie zu drei Meter großen Ninjaschnecken. Sie brechen durch die gläserne Tür und verwüsten springend die Station mit ihrem Säureschleim. Um eine Schnecke in einer Runde zu erreichen, muss dem SC erst ein Agilitätswurf (\TN4) gelingen.

		\nsc{Schnecke}{4}{8}{8}, Agilität+1; Anzahl: SC-1

		\subsection{Szene IV: Invasion}

		\noindent
		SC/2 Aliens haben sich Zutritt in die Station verschafft und ziehen geräuschvoll die Aufmerksamkeit auf sich. Die Laute kommen aus~\ldots

		\tabelle{c X}{
		\thead{W4} & \thead{Beschreibung} \\
		}{
			1 & Paläontologie: Sie verstecken sich im Gerippe eines Tachiosaurierskeletts. \\
			2 & Chemielabor: Sie bewaffnen sich mit blubbernden Phiolen und Explosivschwämmen. \\
			3 & Hydrospeicher: Sie zerstören Hydrotanks. Wasser schwappt auf die Flure. \\
			4 & Klonkammer: Sie versuchen sich (erfolgreich?) im Klon-o-nator zu vervielfältigen. \\
%			5 & Wartungsküche: Sie inspizieren angewidert die Behälter mit Space-Tofu und Astralsoja. \\
%			6 & Astrometrie: Sie schmeißen mit den Planeten und Monden des riesigen Modells des Sonnensystems um sich \\
		}

		\subsection{Szene V: Das Ritual}

		\noindent
		Kurz vor Ende der Sonnenfinsternis vollziehen Alienschamanen vor den Fenstern wild tanzend das \say{Schlund-Fressen-Alles-Auf-Ritual} und rufen ein Schwarzes Loch in der Kommandozentrale herbei, das direkt ins All führt. Sofort fängt es an, alles zu verschlingen, was nicht niet- und nagelfest ist. Wenn sich die Androiden nicht beeilen, wird von der Station nicht mehr viel übrig bleiben. Das Loch mit Nanitenbinden zu stopfen erfordert äußerstes Geschick (langfristige Aktion, \TN30). Wer dabei einen automatischen Fehlschlag würfelt, muss rasch von einem Kollegen gefasst werden (\TN4) oder wird ins Loch gesogen und schwebt nun in der Umlaufbahn Rimols.

	\mysection{Ende gut, alles gut?}

		\noindent
		Die Spieler gewinnen, wenn sie die Sonnenfinsternis heil überstehen. Sie verlieren, wenn alle SC überwunden oder ins All gesogen wurden.
}
