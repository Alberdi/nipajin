% (c) 2009-2015 by Markus Leupold-Löwenthal
% This file is released under CC BY-SA 4.0.

\renewcommand{\celebritiesVersion}{v1.1}

% CHANGELOG-de
%
% 1.1
%   - Tippfehler
% 1.0
%   - Erstfassung

% --- language dependent typography stuff --------------------------------------

\renewcommand{\say}[1]{„\textit{#1}“}
\setdefaultlanguage[spelling=new]{german}

\renewcommand{\fsNormal}{\fontsize{10pt}{12pt plus 0.1pt minus 0pt}}
\renewcommand{\fsSmall}{\fontsize{9pt}{11pt plus 0.1pt minus 0pt}}

% --- pdf metadata -------------------------------------------------------------

\hypersetup{
	pdfsubject={Ein kurzes NIP'AJIN Szenario rund um Promis am Wiener Opernball.},
	pdfkeywords={nipajin, nip'ajin, one-shot, oneshot, shots, Rollenspiel, System, frei, RSP, RPG},
	pdfpagelayout=TwoPageLeft,
}

% --- fine print ---------------------------------------------------------------

\renewcommand{\celebritiesTranslation}{nobody}
\renewcommand{\celebritiesDisclaimer}{(Logos und Marken ausgenommen.)}

% --- main texts ---------------------------------------------------------------

\renewcommand{\celebritiesPlayers}{%
	Ein Szenario für drei oder mehr Charaktere, das ohne Spielleiter auskommt.
}
\renewcommand{\celebritiesSummary}{%
	NIP'AJIN Shots sind kurze Pen-\&-Paper Rollenspielszenarien. Sie sind auf eine Spieldauer von etwa 2 Stunden ausgelegt.

	Um das folgende Szenario spielen zu können, sind die NIP'AJIN Regeln nötig. Diese gibt es kostenlos unter~\ldots
}
\renewcommand{\celebritiesHeadline}{Fiese Promis}
\renewcommand{\celebritiesToc}{Szenario: Fiese Promis}
\renewcommand{\celebritiesText}{%
	{\itshape
		Ein Donnerstag in der zweiten Februarhälfte: Der Opernball, seit über 100 Jahren Höhepunkt der Wiener Ballsaison, lockt fünftausend Gäste in die Staatsoper. Sehen und gesehen werden lautet die Devise der High Society.

		Ihr seid Promis, habt 1.000 Euro für den Eintritt berappt und wisst, dass sich der Medienrummel nur positiv auf eure Karriere auswirkt. Vorausgesetzt, ihr könnt euch gegen die anderen Promis durchsetzen! Eure Waffen: üble Nachrede, Herumgezicke und der Ellbogen am Buffet.

		Sorgt dafür, dass eure Mitspieler schlecht dastehen. Geht Allianzen ein, nur um sie kurz darauf zu brechen. Seid einfach fiese Promis. Aber werdet nicht zu handgreiflich, denn sonst wird man euch des Hauses verweisen.
	}

	\mysection{Charaktere}

		\noindent
		Jeder Spieler erfindet einen Promi mit \HD6, einem Vorteil+1 und einem Nachteil-1. Folgende Personen des öffentlichen Lebens können dabei als Inspirationsquelle dienen, ohne ihnen in der Realität etwas unterstellen zu wollen:

		\keyword{Arnold Schwarzenegger:} Bodybuilder, Filmstar und Exil-Amerikaner.

		\keyword{Niki Lauda:} Ehemaliger F1-Rennfahrer, jetzt Fluglinienbetreiber.

		\keyword{Christina Stürmer:} Junge und international erfolgreiche Pop-Sängerin.

		\keyword{Richard Lugner:} Ehemaliger Bauunternehmer. Überrascht jedes Jahr mit einer anderen internationalen Schönheit an seiner Seite.

		\keyword{Elfriede Jelinek:} Schriftstellerin und Literaturnobelpreisträgerin.

		\keyword{DJ Ötzi:} Entertainer und Schlagersänger.

	\mysection{Setting}

		\noindent
		Das Szenario spielt in der festlich dekorierten Wiener Staatsoper. Die zahlreichen Gäste verteilen sich unter anderem auf folgende Räumlichkeiten:

		\keyword{Foyer}: Längliche Halle und Eingangsbereich im Süden.

		\keyword{Feststiege}: Eine breite Marmortreppe mit mehreren Absätzen führt zu den Obergeschossen und einer Galerie am oberen Ende.

		\keyword{Zuschauerraum}: Der Ballsaal. Bühne, Orchestergraben und  Stühle wurden entfernt.

		\keyword{Marmor- \& Mahlersaal}: Mit Ölbildern, Spiegeln und großen Kronleuchtern geschmückte Säle im Obergeschoss. Zahlreiche Tische gewähren müden Ballbesuchern Rast.

		\keyword{Schwind-Foyer}: \say{Rien ne va plus} -- nichts geht mehr bei Roulette und Poker im Bereich zwischen Marmor- und Mahlersaal.

		\keyword{Balkon}: Zwei Balkone überblicken die Ringstraße. Einer dient als Zigarrenbar, einer als Weinbar.

		Nicht angeführte Räume wie Garderoben, Küchen u.ä. dürfen selbst hinzugefügt werden.

	\mysection{Ablauf}

		\noindent
		Nach der Eröffnungsrede um 22:00, dem traditionellen Spruch \say{Alles Walzer!} und dem ersten Tanz von 150 choreographierten jungen Paaren beginnt die Promi-Schlacht.

		Ziel der Spieler ist es, ihren Promi gut da stehen zu lassen. Dazu wird der Abend des Opernballs nach den Konfliktregeln abgehandelt. Jede Runde dauert 30 Minuten und es kommt meist zu verbalen Attacken. Charaktere, die durchatmen müssen, können ihre Runde nutzen, um sich an einem der Buffets an Würstel und Gulasch zu laben. Sie dürfen dann wie bei einer \emph{Nachtruhe} versuchen, ihren \HD~erhöhen.

		Dieses Szenario benötigt keinen Spielleiter, jeder kann einen Promi spielen. Da niemand NSCs lenkt, sollten sich die Charaktere miteinander statt mit den anderen Gästen befassen.

		Überwundene Charaktere haben vom Opernball die Nase voll und ziehen wutentbrannt ab. Wer übrig bleibt, oder bei der Sperrstunde um 5 Uhr noch den höchsten \HD~hat, gewinnt.
}
