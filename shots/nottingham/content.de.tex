% (c) 2009-2015 by Markus Leupold-Löwenthal
% This file is released under CC BY-SA 4.0. Please do not apply one-way compatible licenses.

\renewcommand{\robinVersion}{v0.1}

% CHANGELOG-de
%
% 0.1
%   - Erstfassung

% --- language dependent typography stuff --------------------------------------

\renewcommand{\fsNormal}{\fontsize{9pt}{11.25pt plus 0.1pt minus 0pt}}
\renewcommand{\fsSmall}{\fontsize{8.5pt}{9.5pt plus 0.1pt minus 0pt}}

% --- pdf metadata -------------------------------------------------------------

\hypersetup{
	pdfsubject={Ein kurzes NIP'AJIN Szenario im Sherwood Forest.},
}

% --- fine print ---------------------------------------------------------------

\renewcommand{\robinCredits}{}

% --- main texts ---------------------------------------------------------------

\renewcommand{\robinPlayers}{%
	Ein Szenario für drei bis fünf Gesetzlose.
}
\renewcommand{\robinHeadline}{Turnier in Nottingham}
\renewcommand{\robinToc}{Szenario: Turnier in Nottingham}
\renewcommand{\robinText}{%
	{\itshape
		England 1186. Während König Henry II. seine Aufmerksamkeit den Spannungen in Frankreich widmet, unterdrückt der Sheriff von Nottingham mit Hilfe der Kirche die bäuerliche Bevölkerung. Wegen minderer Verbrechen angeklagt, seid ihr vor seinem Gesetz in den Sherwood Forest geflohen und habt euch Robin Hoods Bande, den Merry Men, angeschlossen.

		Morgen soll in Nottingham das jährliche Bogenturnier statt finden. Robin hatte vor, verkleidet teilzunehmen und zu gewinnen. Der Sheriff wäre dann verpflichtet, ihm einen Wunsch zu gewähren -- und er würde um die Hand Lady Marians bitten. Doch Robin hat sich an der Hand verletzt und Schwierigkeiten, zu zielen.

		Vor der Kirche als Irrlehrer in den Wald geflohen, unterbreitet Alexander von Neckam die Idee, die Zielscheiben mit Magneten zu manipulieren. Dann wäre nur noch einer der silbernen Turnierpfeile des Sheriffs auszuborgen, um authentische Kopien aus Eisen für Robin anzufertigen. Voilà, die Chancen Robins wären wieder hergestellt. Freiwillige für die Aktion sind rasch gefunden - ihr!

		Im Schutz der Dunkelheit eilt ihr los. Euere Agenda: in den Turnierplatz eindringen sowie die Kirche des St. Nicholas besuchen, wo die Silberpfeile für das morgige Turnier geweiht werden.
	}

	\mysection{Setting}

		\noindent
		Das Szenario spielt in Nottigham, das 1500 Einwohner zählt. Im Westen der Stadt schmiegt sich ein etwa 30 Meter hoher Fels in eine Biegung des Flusses Leen, auf dem die Burg von Nottingham steht. An ihrem Fuße befindet sich eine große, von einem Holzwall umgebene Wiese. Für das Turnier wurden am Areal Tribünen und allerlei Stände aufgebaut. Direkt außerhalb des Walls befindet sich das s.g. normannsiche Viertel, in dem die schlichte Steinkirche des St. Nicholas steht.

	\mysection{Charaktere}

		\noindent
		Die Spieler verkörpern Charaktere mit mittelalterlichen Berufen, \zB~Bäcker, Bauer, Färber, Kerzenzieher, Metzger,  Seiler, Schneider, Steinmetz oder Weber. Aktionen, die zum Beruf passen, sind um +1 erleichtert.

		Die Charaktere tragen drei faustgroße, magnetische Brocken mit sich.

	\mysection{Ablauf}

		\noindent
		Die Charaktere müssen in der Nacht zwei Probleme lösen: die Zielscheiben manipulieren und einen der Pfeile mitgehen lassen. Davon hängen am Turniertag die Chancen Robins ab. Im Szenario könnten es die SC mehrfach mit Wachen des Sheriffs zu tun bekommen. \keyword{Wachen}: \HD3, \AD6, \RD4, Anzahl SC-1.

		\subsection{Die Zielscheiben}

		Das Holztor zum Turnierplatz ist in der Nacht abgesperrt und wegen dem bevorstehenden Turnier bewacht. Der Wall ist mit Fackeln gut ausgeleuchtet, auf dem Areal dahinter ist es jedoch dunkel.

		Fünf Zielscheiben sind vor einer Burgmauer aufgestellt. Um zu ihnen zu gelangen, ist der Wall zu überwinden. Die Charaktere könnten dazu \zB~für Ablenkung sorgen und klettern, mit einer List die Wachen dazu bringen, sie in das Areal zu lassen, oder diese einfach niederringen.

		Bei den mannshohen Strohzielscheiben angekommen, kann versucht werden, die Magnete in diese einzuarbeiten. Dabei ist rasch aber sorgfältig vorzugehen (langfristige Aktion, \TN10 pro Zielscheibe, 1 Minute/Runde, -2 wegen Dunkelheit). Am Ende jeder zweiten Runde taucht eine Komplikation auf: neugirige Kinder, laute Ziegen, ein betrunken singender Arbeiter oder ein hinter Stohballen beschäftigtes Liebespaar. Wird eine Komplikation nicht abgewimmelt (Gruppenaktion\TN4), kommen Wachen Nachschau halten und müssen ruhig gestellt werden.

		\subsection{Die Pfeile}

		Die Silberpfeile des Sheriffs werden über Nacht in der Kirche von St. Nicholas gesegnet und dort am Altar in einer  Schatulle aufbewahrt. Zum Schutz von Kirche und Pfeilen patrouliert ein Wachtrupp rund um das Steingebäude, an denen es vorbeizukommen gilt. Vorzugeben, lediglich beten zu wollen, erleichtert das (+1).

		Im Inneren der Kirche beten Priester die ganze Nacht über. Zum Altar zu gelangen ist nicht schwierig, doch die Schatulle zu öffnen oder gar einen Pfeil zu entwenden werden die Priester nicht zulassen und ggf. die Wachen rufen, wenn sie nicht abgelenkt werden.

		Gottesfürchtige Charaktere werden in einer Kirche nichts stehlen und \zB~nur einen Lehm-Abdruck eines Pfeiles oder eine Skizze anfertigen. Das wird den Herrn beim Turnier milde stimmen. Lassen die Charaktere alle Pfeile mitgehen, wird das Turnier abgesagt und das Szenario endet vorzeitig.

		\subsection{Das Turnier}

		Mit Horngeblase startet am nächsten Tag das Turnier. Acht Schützen nehmen Teil, darunter der, mit einem falschen Bart verkleidete Robin Hood, der sich als Monsieur Chapeau ausgibt. Der Sheriff betrachtet das Schauspiel von einer Tribüne, Marian sieht aus der Ferne vom Fenster ihres Turmzimmers zu. Die Charaktere stehen unter den Zuschauern.

		Die Spieler würfeln für Robin (\HD6,\AD8,\RD6). Jede Runde muss er den, ihm ausgehändigten Pfeil unauffällig tauschen (\TN4, Ablenkungsmanöver der Charaktere erleichtern das um +1), bekommt zufällig eine der Zielscheiben zugeordnet (was Charaktere ebenso beeinflussen versuchen können) und schießt mit seinem \AD8. Robin erhält +1 göttlichen Beistand, wenn in der Kirche nicht gesündigt wurde und bei Einsatz eines Eisenpfeils +2 pro Magnet / +1 pro Magnetfragment in der Zielscheibe. Es gibt drei Runden, der beste Gegner wird auf eine Summe von 15 kommen, die Robin überbieten muss.

		\subsection{Plan B}

		Geht beim Turnier etwas schief, lässt Robin lachend seine Tarnung fallen, beleidigt den Sheriff, und versucht, über einen, provisorisch aufgestellen Holzturm auf die Burgmauer zu kommen. Er ruft den Charakteren \say{Verschafft mir Zeit!} und eilt einem noch unschlüssigen Wachtrupp davon (Anzahl SC+1).

		Es kommt zum Kampf. Werden die Wachen überwunden, gibt das Robin genug Zeit, mit Marian -- für alle Anwesenden bestens sichtbar -- entlang der Burgmauer zu fliehen und mit ihr von dort in den Leen zu springen.

	\mysection{Ende gut, alles gut?}

		\noindent
		Das Szenario endet erfolgreich, wenn Robin mit seiner Geliebten Nottingham verlässt -- als Gewinner oder mit ihr fliehend. Werden Charaktere im Szenario überwunden, landen sie in Gefangenschaft im Kerker von Nottingham.
}
