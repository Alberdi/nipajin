% (c) 2009-2016 by Markus Leupold-Löwenthal
% This file is released under CC BY-SA 4.0. Please do not apply one-way compatible licenses.

\renewcommand{\metashotVersion}{v1.0.1}

% CHANGELOG-de
%
% 1.0.1
%   - Lektorat Onno
% 1.0
%   - Erstfassung

% --- language dependent typography stuff --------------------------------------

\renewcommand{\fsNormal}{\fontsize{9pt}{11.25pt plus 0.1pt minus 0pt}}
\renewcommand{\fsSmall}{\fontsize{8.5pt}{9.5pt plus 0.1pt minus 0pt}}

% --- pdf metadata -------------------------------------------------------------

\hypersetup{
	pdfsubject={Ein Leitfanden für \nipajin\ Shots Autoren.},
}

% --- fine print ---------------------------------------------------------------

\renewcommand{\metashotCredits}{
	Text:~Markus Leupold-Löwenthal; Lektorat:~Onno Tasler%
}

% --- main texts ---------------------------------------------------------------

\renewcommand{\shotSummary}{%
	\nipajin\ Shots sind kurze Pen-\&-Paper Rollenspielszenarien. Sie sind auf eine Spieldauer von 1--2 Stunden ausgelegt.

	Dieser Leitfaden soll dir helfen, eigene Shots zu verfassen. Die dazu nötigen Regeln gibt es kostenlos unter:
}

\renewcommand{\metashotPlayers}{%
%	Ein Leitfaden für Rollenspielautoren.
}
\renewcommand{\metashotHeadline}{Meta-Shot}
\renewcommand{\metashotToc}{Leitfaden: Meta-Shot}
\renewcommand{\metashotText}{%

	\noindent
	Hallo! Schön, dass du einen \nipajin\ Shot verfassen möchtest.

	\mysection{Was sind Shots?}

		\noindent
		Oberflächlich betrachtet sind \nipajin\ Shots Abenteuer von 6000 Zeichen bzw. 900 Worten Länge in Schriftgröße 9 auf zwei A5-Seiten -- die selbe Länge hat dieser Leitfaden. Aber ein Shot ist mehr als das: Er hat einen gewissen Aufbau und sollte i.d.R. folgende Eigenschaften aufweisen:

		\keyword{Direkt ins Geschehen:} Ein Shot konfrontiert Spieler bereits in einem Prolog mit einer konkreten Aufgabe und führt sofort in die Handlung. Die Charaktere müssen nicht erst einen Auftraggeber suchen oder von ihm überzeugt werden.

		\keyword{Beispielcharaktere:} Jeder Shot enthält Beispielcharaktere, die einander ergänzen und die Handlung fördern.

		\keyword{Ein Kracher zu Beginn und zum Schluss:} Damit die Spieler gleich aktiv werden, sollte die erste Szene voller Action sein: Das Piratenschiff eröffnet das Feuer! Die Hängebrücke droht zu reißen! Ninja-Gorillas greifen an! Ebenso sollte eine actiongeladene Schlussszene die Spieler mit dem Gefühl zurücklassen, etwas erreicht zuhaben.

		\keyword{Messbare Fortschritte:} Um weniger erfahrenen Spielern zu helfen, hat jede Szene ein spieltechnisches Ende. Diese Vorgabe sollte aber auch alternative Herangehensweisen zulassen.

		\keyword{Zwei Stunden Spielzeit:} Da Shots nicht nur, aber auch, von Einsteigern und an Orten mit Zeitdruck -- etwa Conventions -- gespielt werden, haben sie etwa die Dauer eines Brettspiels.

		\keyword{Begnüge dich mit Grundregeln:} Führe wenig bis keine neuen Regel-Subsysteme ein. Kleine Besonderheiten, wie die Bonbon-Regel von \say{Mit \nipajin\ Allein Daheim,} können aber genutzt werden, um Shots etwas Besonderes zu geben.

	\mysection{Was sind keine Shots?}

		\noindent
		Nicht so einfach machbar sind erfahrungsgemäß~\ldots

		\keyword{Ergebnisoffene Szenarien}: Eine Sandbox mit Orten und Personen eignet sich mangels konkreter Aufgabe nicht besonders für das Shots-Format.

		\keyword{Investigative Szenarien}: Action wirkt hier oft gestellt, die Spielzeit ist schwer abzuschätzen und der Fortschritt schwierig zu messen.

		\keyword{Settings mit Erklärungsbedarf}: Du hast nicht den Platz, die Kultur von originellen Wuschelkopfelfen mit Fußpilzfetisch zu schildern. Bleib bei gängigen Klischees -- darunter kann sich der Leser auch mit wenigen Worten etwas vorstellen.

		\keyword{Epische Kampagnen}: Shots sind halbe Abenteuer -- die andere Hälfte haben die Charaktere bereits im Prolog erlebt. Lass die Charaktere ein örtliches Problem lösen, statt sie mit einem Ring um die ganze Welt zu schicken.

	\mysection{Aufbau}

		\noindent
		Der erste Satz sollte die \keyword{Zahl der Charaktere} nennen, für die der Shot verfasst ist. Sei dabei realistisch: Meinst du wirklich \say{2--6} oder passen sechs Personen gar nicht in den Fluchtwagen? Schwankt die Schwierigkeit zu sehr? Es ist O.K., wenn du dich \zB~ auf \say{4--5} festlegst. [10 Worte]

		Der kursiv gesetzte \keyword{Prolog} kann vom SL bedenkenlos vorgelesen werden und sagt den Spielern alles, was sie wissen müssen -- inklusive Setting-Infos. Nimm ihnen die Auftragsfindung ab und konfrontiere sie mit dem konkreten Problem, das es in Szene 1 zu lösen gilt. [250 Worte]

		Lasse dem Prolog eine Tabelle mit \keyword{Beispielcharakteren} folgen. Gib dazu bloß Namen, Motiv (ein kurzer Satz), Vor- und Nachteil an -- für mehr ist nicht Platz. Achte darauf, dass sich die Fähigkeiten der Charaktere unterscheiden und sie einander ergänzen, d.h. die Stärke eines SCs gleicht die Schwäche eines anderen aus. [75 Worte]

		Den Beispielcharakteren folgen die Informationen zum \keyword{Setting}. Wenn der SL das Geschriebene sofort den Spielern erklären muss, packe es gleich mit in den Prolog. Den liest der SL nämlich auch. [200 Worte]

		Schildere jetzt den \keyword{Ablauf} des Szenarios, indem du die Handlung in Szenen teilst. Für die angepeilte Spielzeit genügen zwei bis vier. Mit einem einfachen Kniff kannst du die Illusion einer längeren Geschichte erreichen: schildere ein, zwei Szenen bereits im Prolog und wie deren Ausgang zur ersten gespielten Szene führte. Sieh zu, dass jede Szene ein messbares Ende hat. In einem Konflikt sind Gegner mit den Werten X und Y zu besiegen, Informationsbeschaffung ist eine langfristige Aktion mit Zielwert Z, ein Stadtrat wird durch Aktionen A, B oder C überzeugt. [250-350 Worte]

		Wenn du auf kleine \keyword{Szenarioregeln} nicht verzichten magst, erkläre diese jetzt. [0-100 Worte]

		Die Überschrift \keyword{Ende gut, alles Gut?} leitet das Schlusswort ein, das dem SL sagt, ob das Szenario gut oder schlecht endet. Falls noch Platz bleibt, kannst du in einem kurzen Ausblick Ideen geben, wie das Szenario fortgesetzt werden könnte. [50 Worte]

	\mysection{Regelhilfe}

		\noindent
		Dieser Abschnitt soll dir helfen, die Schwierigkeiten der Szenen zu dimensionieren.

		\keyword{Gruppenaktionen} binden mehrere Spieler ein und sind die empfohlene Variante, Erfolge zu messen. Mach den Fortgang des Szenarios auf keinem Fall von einem einzigen Wurf abhängig, der scheitern könnte! Szenarien für weniger Charaktere benötigen niedrigere \TN\ -- sie haben geringere Chancen:

		\tabelle{X c c c c}{
			\thead{Schwierigkeit} & \thead{\TN} & \thead{1 SC} & \thead{2--3 SC}  & \thead{4--6 SC} \\
		}{
			einfach		  & 2 & 88\% & 90\%	 & 91\%	 \\
			durchschnittlich & 4 & 65\% & 80--85\% & 90\%	 \\
			schwer		   & 6 & 40\% & 60--70\% & 80--85\% \\
			meisterlich	  & 8 & 30\% & 50--60\% & 65--75\% \\
		}

		\noindent
		Beschreibe Gruppenaktionen im Text \zB~mit \say{Um den Banditen zu folgen, müssen die Charaktere der Spur folgen (Gruppenaktion, \TN6).}

		\keyword{Langfristige Aktionen} sind ideal, um den Fortschritt in konfliktlosen Szenen zu messen -- sie werden früher oder später geschafft. Ihre Länge hängt von der Zahl der Charaktere und dem \TN\ ab:

		\tabelle{X c c c c c c c}{
			\thead{Länge in Runden} & \thead{\TN} & \thead{1 SC} & \thead{2--3 SC} & \thead{4-6 SC} \\
		}{
			kurz	  & 10 &  $\sim$3 & $\sim$2 & $\sim$2 \\
			mittel	  & 20 &  $\sim$5 & $\sim$4 & $\sim$3 \\
			lang	  & 30 &  $\sim$7 & $\sim$6 & $\sim$5 \\
			sehr lang & 40 &  $\sim$9 & $\sim$7 & $\sim$6 \\
		}

		\noindent
		Beschreibe langfristige Aktionen \zB~mit \say{Die Charaktere könnten versuchen, das Boot zu reparieren (langfristige Aktion, \TN30, 10 Minuten pro Runde).}

		Faire \keyword{Konflikte} sind in \nipajin\ sehr komplex zu berechnen. Als Faustregel kann jeder Gegner mit einem Wert versehen werden (\say{S} ist sein jeweiliger Waffenschaden, z.B. 1 oder 2):

		\vspace{3pt}
		\centerline{\keyword{SC} = 16 + S³}
		\vspace{3pt}
		\centerline{\keyword{NSC} = \HD + \AD/2 + \RD/2 + S³}
		\vspace{3pt}

		\noindent
		Die Summe aller NSC wird dann mit jener der SC verglichen:

		\tabelle{X c c c c}{
			\thead{NSC / SC} & \thead{bis 25\%} & \thead{bis 50\%} & \thead{bis 75\%} & \thead{> 75\%}  \\
		}{
			Schwierigkeit & einfach & normal & schwer & tödlich \\
		}

		\noindent
		In Szenarien ohne Heilung können drei einfache, zwei normale oder ein schwerer Konflikt bewältigt werden. Tödliche Konflikte sind ohne Hilfe extrem riskant bis aussichtslos. Beschreibe Gegner so:

		\nsc{Name}{?}{?}{?}, gut+X, schlecht-Y, Anzahl: Z

	\mysection{Ende gut, alles gut?}

		\noindent
		Hoffentlich hat dich dieser Leitfaden nicht abgeschreckt. Vorlagen, Beispiele und die \nipajin-Symbolschrift findest du unter:  \href{https://github.com/ludus-leonis}{\emph{github.com/ludus-leonis}}

}
