% (c) 2009-2015 by Markus Leupold-Löwenthal
% Initial translation provided by Szymon Noobirus Piecha
% This file is released under CC BY-SA 4.0.

\renewcommand{\nipajinVersion}{v1.5.2-pl}

% CHANGELOG-de
%
% 1.5.1
%   - minor typos
% 1.5
%   - translation by Szymon Piecha

% --- redefine typography stuff to match language

\renewcommand{\say}[1]{„\textit{#1}”}
\setdefaultlanguage{polish}
\newcommand{\fontsizenormal}{\fontsize{10.5pt}{12pt plus 0.1pt minus 0pt}}
\newcommand{\fontsizesmall}{\fontsize{9pt}{10pt plus 0.1pt minus 0pt}}

% --- pdf metadata -----------------------------------------------------

\hypersetup{
	pdftitle={NIP'AJIN},
	pdfauthor={Markus Leupold-Loewenthal},
	pdfsubject={Nieskomplikowany, uniwersalny system RPG.},
	pdfkeywords={nipajin, nip'ajin, system, uniwersalny, RPG}
}
\renewcommand{\nipajinTranslation}{Tłumaczenie: Szymon \say{Noobirus} Piecha}

% --- headlines --------------------------------------------------------

\renewcommand{\nipajinDisclaimer}{(Oprócz logo i~nazwy marek.)}

% --- main texts -------------------------------------------------------

\renewcommand{\HD}{KO}
\renewcommand{\AD}{KA}
\renewcommand{\RD}{KR}
\renewcommand{\PT}{PR}
\renewcommand{\FT}{CD}

\renewcommand{\nipajinTextIntro}{%
	Nieskomplikowany, uniwersalny system RPG od \ludusleonis.

	\nipajin\ nadaje się na pojedyncze przygody i~krótkie kampanie. Ten niemiecki akronim wymawia się \say{nip-a-czin} i oznacza \say{nikt nie jest doskonały, lecz każdy przydatny}. Każdy gracz ma tutaj swoje 5 minut chwały i~nikt nie jest ograniczony mechaniką.
}

\newcommand{\nipajinTableModifier}{%
	\tabelle{X c}{
	\thead{Znajomość} & \thead{Modyfik.} \\
	}{
	poważna słabość                      & -4 \\
	niedoświadczony                      & -2 \\
	trochę się zapomniało                & -1 \\
	średni                               &  0 \\
	trochę wiedzy, hobby                 & +1 \\
	wieloletnie doświadczenie, rutyna    & +2 \\
	dziesiątki lat dośw./ weteran        & +4 \\
	}
}

\newcommand{\nipajinTableTargets}{%
	\tabelle{l c X}{
	\thead{Trudność} & \thead{\TN} & \thead{Przykład} \\
	}{
	łatwy          & 2 & -- \\
	dobre warunki  & 3 & dobre narzędzia \\
	zwyczajny      & 4 & -- \\
	złe warunki    & 5 & brak światła \\
	ciężki         & 6 & żonglerka nożami \\
	mistrzowski    & 8 & spacer po linie \\
	legendarny     & 12 & -- \\
	}
}

\renewcommand{\nipajinHeadlinePlayer}{Zasady dla graczy}
\renewcommand{\nipajinTocPlayer}{Zasady dla graczy}
\renewcommand{\nipajinTextPlayer}{%
	\noindent
	Każda \keyword{postać} zaczyna jako pusta kartka A4. Ta \keyword{karta postaci} zostaje podzielona linią na dwie karty A5, później prawa strona na dwie kartki A6.

	Teraz gracze zapisują ogólne informacje, jak imię, pochodzenie i~wygląd na lewej kartce a następnie \keyword{tło} postaci. Można je zapisać za pomocą słów-kluczy lub opisu. W~opisie należy ustalić co postać dotychczas zrobiła, a~nie to, co potrafi robić. To ostatnie ustali się w~grze. Na karcie postaci stoi np. \say{nosił fortepiany} zamiast \say{jest silny}. \keyword{Uzbrojenie} (\refPage{labelEquipment}) i~ \keyword{efekty} (\refPage{labelEffects}) zawierają to, co gracze i~mistrzowie uważają za pasujące.

	Na koniec kładziemy kości k4, k6, k8, k10 i~k12 na prawą górną kartkę. Gracz wybiera jedną z tych kostek jako \keyword{kość oporu} (\HD)  i~kładzie ją na lewą stronę kartki z jedynką u góry. Jeśli granica kostki zostanie przekroczona (np. siódemka w~k6), postać wypada z~gry.

	\mysection[labelTaskresolution]{Kostki}

		\noindent
		Dopóki nie ma wątpliwości, gracze i~mistrzowie spokojnie rozwijają akcję gry. Lecz jeśli w~czasie ważnej \keyword{akcji} narodzi się pytanie, czy postaci uda się coś zrobić, wtedy gracz wybiera \keyword{dostępną kość} z~górnej ćwiartki karty i~rzuca. Jeśli wypadnie jedynka, akcja kończy się \keyword{automatyczną porażką} w~innym przypadku, dodaje się do rzutu \keyword{modyfikator} ze \keyword{znajomości} bohatera. Znajomość jest ustalana przez MG, na podstawie tła postaci.

		\nipajinTableModifier

		\noindent
		Akcja jest udana, jeśli wynik osiągnie lub przekroczy \keyword{cel}\,\TN\ wyznaczony przez prowadzącego. Wyliczona jedynka, w~przeciwieństwie do wyrzuconej jedynki, nie jest automatyczną porażką, ale rzadko wystarczy.

		\nipajinTableTargets

		\noindent
		Po rzucie, \keyword{zużyta kostka} ląduje na dolnej ćwiartce karty postaci. Gdy \emph{wszystkie} zostaną zużyte -- i~tylko wtedy! -- gracz może dać swojej postaci \keyword{odsapnąć}, zanim kostki wylądują znowu na górę: musi przeczekać całą rundę w konflikcie.

		Raz przydzielone modyfikatory zapisywane są na karcie postaci i o ile nie zajdą jakieś specjalne okoliczności, są one stałe.

		Nie można używać kości, które nie mają szans by zaliczyć akcję. Cel \TN\ akcji zwiększa się o~1 z~każdą próbą, jeśli postać chce \keyword{powtórzyć} nieudany test współgracza.

	\mysection[labelConflict]{Konflikty}

		\noindent
		Konflikty podzielone są na \keyword{rundy}, których długość określa prowadzący. W~każdej rundzie postać może wykonać jedną akcję, np. atakować i~może reagować na akcje wroga, np. parować. \HD\ przeciwników musi zostać pokonana. Za pomocą siły lub bez.

		Na początku każdej rundy gracze wybierają jednocześnie \keyword{kość akcji} (\AD) i~jedną \keyword{kość reakcji} (\RD) z~puli dostępnych kości. Za ich pomocą, rozwiązują wszystkie akcje i~reakcje w~tej rundzie. Własnowolnie, lub gdy nie ma dość kości, można zrezygnować z~\AD\ lub \RD\ i~tym samym zrezygnować z~wykonywania akcji lub reakcji. Jeśli ktoś nie ma \emph{żadnych} dostępnych kości, musi przeczekać i~odsapnąć.

		Wybrana \AD\ decyduje też o~\keyword{inicjatywie} w~rundzie. Mniejsze kości działają przed większymi (np. k6 przed k8) w~przypadku remisu, wykonuje się rzut kością.

		By atak się udał, wynik \AD\ musi być większy od \RD, wyrzucenie jedynki jest zawsze porażką. Przy udanym ataku, reagujący otrzymuje \keyword{ranę} i~podnosi swoją \HD\ w~górę. Nawet akcje bez użycia siły pomagają przełamać opór: powodują \keyword{traumę}, którą liczy się znacznikami lub za pomocą dodatkowej k20. Jeśli suma traumy i~obecnego wyniku \HD\ przekroczy \HD, przeciwnik jest pokonany (przestraszony, zirytowany,~\ldots).

		Za pomocą \keyword{ataku obszarowego} postać może zaatakować wielu wrogów (np. wir ciosów, zastraszenie grupy, kula ognia,~\ldots). Za każdy dodatkowy cel, wynik rzutu zostaje obniżony o~\makebox{-2}. Każdy przeciwnik może osobno zareagować.

		Postać może też kogoś \keyword{osłaniać}, jeśli nie wybierze w~tej rundzie \AD. Może przyjmować na siebie ataki wszystkich w~zasięgu, z~połową swojej \RD\, np. dwa w~k4 lub trzy przy k6. Osłaniający otrzymuje rany, jeśli nie uda mu się ta akcja.

	\mysection[labelEquipment]{Ekwipunek}

		\noindent
		Nie ma tabeli z~przedmiotami. Normalne bronie zadają jedną ranę przy celnym ataku, specjalne lub magiczne dwie a~ palne lub eksplodujące trzy do czterech. Improwizowane uzbrojenie dostaje \makebox{-1} do \AD. Zbroje, w~zależności od wykonania, zwiększają \RD\ o~+1 lub +2.

	\mysection[labelHeal]{Leczenie}

		\noindent
		Po każdej \keyword{przespanej nocy} postacie \say{odsapują}. Dodatkowo można spróbować zmniejszyć rany na \HD. Gracz zapamiętuje wartość i~rzuca \HD. Jeśli wyrzuci mniej, jedna rana się leczy. Więc \HD\ zostaje odłożona z~wynikiem mniejszym o~jeden.

		Prowadzący decyduje, jak szybko leczą się \keyword{traumy}. Zastraszenia zostają zapomniane pod koniec sceny, fobie, klątwy, itp. nękają postać dniami a~nawet tygodniami.
}

\renewcommand{\nipajinHeadlineEffects}{Efekty}
\renewcommand{\nipajinTocEffects}{Efekty}
\renewcommand{\nipajinTextEffects}{\zlabel{labelEffects}%
	\noindent
	Magia, cuda, supermoce czy też moce psioniczne są nazywane \keyword{efektami} i~postać może je już posiadać na początku gry. \nipajin opisuje moce tego typu następująco:

	W czasie \keyword{przygotowania} (\PT) postać wymawia inkantacje lub gestykuluje. Jeśli czas jest \keyword{różny}, to zostaje określony przez gracza. Wykonuje się rzut z~ew. karą \makebox{-2} za każdy dodatkowy cel w~ataku obszarowym. Cel wykonuje rzut na reakcję, by w~pełni uniknąć efektu. Efekty, które nie mają na nikogo wpływać, mają wartość celową ustaloną przez prowadzącego. Jeśli rzut się uda, efekt się utrzymuje, dopóki postać się koncentruje przez \keyword{czas działania} (\FT).

	\keyword{Efekty bliskiego zasięgu} kaleczą jak bronie białe, np. \emph{lodowy dotyk} lub \emph{widmowy miecz}. Trafienie zadaje jedną ranę. \PT:~1~Runda; \FT:~$\infty$

	\keyword{Efekty dalekiego zasięgu} kaleczą jak bronie dystansowe i~do użycia potrzebują jakiegoś ograniczonego materiału lub źródła, np. \emph{magiczny pocisk} potrzebuje prochu strzelniczego. \PT:~1~Runda; \FT:~$\infty$

	\keyword{Efekty nokautujące} unieszkodliwiają wrogów, np. \emph{uśpienie}, \emph{strach}, \emph{zamiana w~kamień} lub \emph{odesłanie}. \PT:~różny; \FT:~ustalony \PT

	\keyword{Efekty wspomagające} pomagają istocie lub polepszają coś, np. \emph{odporność na ogień}, \emph{szybowanie} lub \emph{światło}. \PT:~różny; \FT:~ustalony \PT

	\keyword{Przemiany} formują lub poruszają martwą materią istoty lub jej emocje, np. \emph{woda w~wino}, \emph{zaprzyjaźnienie} lub \emph{telekineza}. \PT:~różny; \FT:~ustalony \PT

	\keyword{Iluzje} wpływają na zmysły istot, np. \emph{tęczowe kolory}, \emph{wywołanie dźwięku} lub \emph{niewidzialność}. \PT:~różny; \FT:~ustalony \PT

	\keyword{Natchnienia} dają wiedzę na temat różnych zagadnień, np. \emph{wykrycie magii}, \emph{czytanie myśli} lub \emph{jasnowidzenie}. \PT:~minuta jeśli chodzi o~teraźniejszość, godzina jeśli przeszłość, dzień jeśli przyszłość; \FT:~--

	\keyword{Uzdrowienia} zamykają rany lub leczą choroby. \PT:~godzina na każdą ranę, dzień na każdą chorobę; \FT:~$\infty$
}

\renewcommand{\nipajinHeadlineGM}{Zasady dla prowadzących}
\renewcommand{\nipajinTocGM}{Zasady dla prowadzących}
\renewcommand{\nipajinTextGM}{%
	\mysection[labelPCs]{Tła}

		\noindent
		Największy nacisk \nipajin\ kładzie na tła bohaterów graczy (BG). Dobre historie postaci obejmują dzieciństwo, wyszkolenie i~co BG robił przez ostatnie lata. Parę ważnych wydarzeń świetnie uzupełnia opis. Powinno się również zapisać wiek i~wygląd.

		To jak ma wyglądać opis, zależy w~głównej mierze od stylu gry. Prowadzący musi tylko zadbać o~to, by z~opisów można było wyczytać jakie silne i~słabe strony ma postać, ponieważ na ich bazie będzie się później decydować, czy jakaś akcja będzie łatwa czy trudna. Luki w~opisach powinny być, przy wcześniejszej rozmowie z~graczem, natychmiast uzupełnione.

	\mysection[labelGroups]{Testy grupowe}

		\noindent
		Jeśli więcej BG bierze udział w~jednej akcji, wykonuje się \keyword{test grupowy}. Każdy z~graczy wybiera sobie kostkę i~wybiera się przywódcę. Przywódca rzuca pierwszy i~decyduje, czy jego rzut liczy się za wszystkich. Jeśli nie, przekazuje przywództwo innemu graczowi itd. Każdy rzut \emph{zastępuje} poprzedni. Wyrzucenie jedynki udaremnia wszystkim próby. Tylko rzucone kości zostają zużyte.

		W czasie wykonywania grupowego rzutu w~konflikcie, grupa dostaje najgorszą inicjatywę. Wróg odpiera wynik ataku swoją \RD\, jeśli atak się uda, dostaje tyle obrażeń, ile każda postać w~grupie mogłaby zadać osobno.

		\keyword{Złożone akcje} mają wysoką wartość celową, np. Naprawa\TN20. Zaangażowani BG pracują w~etapach -- pojedynczych rzutach grupowych, w~których się tylko \emph{zastępuje} -- ku tej sumie. Prowadzący decyduje, ile czasu mija po każdym etapie, np. jeden dzień. Wyrzucona jedynka udaremnia etap, ale nie całą akcję złożoną.

	\mysection[labelNSCs]{Bohaterowie niezależni}

		\noindent
		Prowadzący będzie przedstawiał graczom kolejnych wrogów jak i~przyjaciół. Można ich stworzyć na dwa sposoby.

		\keyword{Główni bohaterowie niezależni} (BN) przygody mogą, tak jak postacie, posiadać tło fabularne, które zawiera słabe i~silne strony. Nie należy jednak z~tym przesadzać, nie dawaj więcej niż dwóm BN obszernej historii. Prowadzący korzysta z~zasad działu dla graczy, jeśli gracze mają konflikt z~taką postacią.

		Większość BN ma tylko \keyword{rolę poboczną}. Są to różni informatorzy, sprzedawcy na bazarze lub nieznośny potwór. By prowadzącemu było łatwiej, takie postacie buduje się na prostszych zasadach. Poza wyglądem i~zachowaniem, wystarczy określić jeszcze \HD, \emph{jedną} \AD\ i~\emph{jedną} \RD. Do tego zostaje zanotowanych kilka słabych i~silnych stron. Przy \HD\ można zapisać wszystko od 1 do $\infty$, przy \AD/\RD\ kostki k2, k3, k4, k5, k6, k8, k10, k12 i~k20. Traumy są liczone bezpośrednio na \HD\ a~nie osobno. Poboczne role mają tę przewagę, że nie muszą odsapnąć, ponieważ \AD/\RD\ się im nie kończy. By było uczciwie, nie powinny być za wysokie.

	\mysection[labelBestiary]{Bestiariusz}

		\noindent
		Poniższe kreatury są jedynie przykładem, nie sugerują, by prowadzić \nipajin\ w~tych klimatach.

		\tabelle{p{1.2cm} c c c X}{
			\thead{Potwór} & \thead{\HD} & \thead{\AD} & \thead{\RD} & \thead{Zdolności}  \\
		}{
			D. szczur &  1 &  2 &  3 & Bieg+4, Ukrywanie+2 \\
			Goblin    &  3 &  4 &  4 & Percepcja+1 \\
			Ork       &  6 &  6 &  6 & Zastraszanie+1, Walka+1, Umysł-1 \\
			Troll     & 10 &  8 &  6 & Walka+2, regeneruje ranę/rundę \\
			Gigant    & 20 &  8 &  8 & Walka+2, Siła+4 \\
			Smok      & 40 & 12 & 10 & Zionięcie+4, odporny na nokautujące \\
		}

	\mysection[labelXP]{Doświadczenie}

		\noindent
		BG określają \say{poziom} otoczenia, jeśli są oni zwyczajnymi podróżnikami, to podane w~powyższej tabeli cechy są dobrymi wytycznymi dla \say{reszty}. Jeśli są goblinami, które odpierają stada \say{bohaterów} atakujących obóz, to \say{bohaterowie} są być może dla graczy silni jak trolle. Prowadzący powinien patrzeć na BG i~BN relatywnie.

		\nipajin\ nie jest przeznaczony do tego, by rozwijać bohaterów w~długich kampaniach. Gdy w~czasie przygody postać długo trenowała, prowadzący może podnieść jedną znajomość postaci na wyższy poziom. Jeśli jednak BG dostaną supermoce, zaleca się nie polepszać \emph{postaci}, lecz osłabić \emph{świat} wokół nich!
}
