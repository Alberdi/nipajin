% (c) 2009-2015 by Markus Leupold-Löwenthal
% This file is released under CC BY-SA 4.0.

\renewcommand{\versionRules}{v1.7-de}

% CHANGELOG-de
%
% 1.7
%   - Hintergrund -> Vorgeschichte
%   - Mehrfachangriff
%   - Flächenangriff umformuliert
%   - Vor/Nachteile von Beispielcharakteren
%   - Überraschte Gegner im Konflikt
% 1.6
%   - WW zählt runter
%   - Klarstellung der Expertise bei "keine Chance auf Erfolg"
%   - Umbenennung Gruppenwurf auf Gruppenaktion und Klarstellung
%   - nur mehr vereinfachte NSCs
% 1.5
%   - rotes layout
%   - multi-language
% 1.4.2
%   - Neues Buildsystem
%   - Layout / Logos neu
% 1.4.1
%   - Klarstellung Gruppenwurf während langfristigen Aktionen
% 1.4
%   - SL-Kapitel neu angeordnet
%   - Gruppenwürfe neu geregelt und in den SL-Teil verfrachtet
%   - LL-Band am Cover
%   - nachbessern -> wiederholen
% 1.3
%   - Lektorat Onno
%   - Gruppenwurf im und außerhalb des Konflikts vereinheitlicht.
%   - Crop
% 1.2
%   - Beginn des Changelogs

% --- language dependent typography stuff ------------------------------

\renewcommand{\say}[1]{„\textit{#1}“}
\setdefaultlanguage[spelling=new]{german}

\renewcommand{\fsNormal}{\fontsize{9.75pt}{11.25pt plus 0.1pt minus 0pt}}
\renewcommand{\fsSmall}{\fontsize{8.5pt}{9.5pt plus 0.1pt minus 0pt}}

\hyphenation{Rol-len-spiel}
\hyphenation{Blatt-hälfte}

% --- pdf metadata & stuff ---------------------------------------------

\hypersetup{
	pdftitle={NIP'AJIN},
	pdfauthor={Markus Leupold-Loewenthal},
	pdfsubject={Ein leichtgewichtiges, freies Rollenspielsystem.},
	pdfkeywords={nipajin, nip'ajin, Rollenspiel, System, frei, RSP, RPG}
}
\renewcommand{\translation}{nobody}

\renewcommand{\backgroundlayername}{Hintergrund}

% --- headlines --------------------------------------------------------

\renewcommand{\disclaimer}{(Logos und Marken ausgenommen.)}
\renewcommand{\headlinePlayer}{\nipajin~für Spieler}
\renewcommand{\tocPlayer}{\nipajin~für Spieler}
\renewcommand{\headlineGM}{\nipajin~für Spielleiter}
\renewcommand{\tocGM}{\nipajin~für Spielleiter}
\renewcommand{\headlineEffects}{Effekte}
\renewcommand{\tocEffects}{Effekte}

% --- language macros --------------------------------------------------

\newcommand{\zB}{z.\,B.}
\newcommand{\uU}{u.\,U.}

% --- main texts -------------------------------------------------------

\renewcommand{\textIntro}{%
	Ein unkompliziertes, universelles Rollenspielsystem von \ludusleonis.

	\nipajin\ eignet sich für Einzelabenteuer und Kurzkampagnen, wird \say{nip-atschin} ausgesprochen und ist ein Akronym für \say{Niemand ist perfekt, aber jeder irgendwie nützlich}. Es sorgt für verteiltes Rampenlicht, ohne die Charaktere in ein enges Regelkorsett zu zwingen.
}

\newcommand{\tableModifier}{%
	\tabelle{X c}{
		\thead{Expertise} & \thead{Modifik.} \\
	}{
		veritable Schwäche                   & -4 \\
		unerfahren, sehr ungeschickt         & -2 \\
		etwas eingerostet                    & -1 \\
		durchschnittlich gut                 &  0 \\
		ein wenig Übung, Hobby               & +1 \\
		jahrelange Erfahrung, Beruf, Routine & +2 \\
		jahrzehntelange Erfahrung, Veteran   & +4 \\
	}
}

\newcommand{\tableTargets}{%
	\tabelle{l c X}{
		\thead{Schwierigkeit} & \thead{\TN} & \thead{Beispiel} \\
	}{
		einfach               & 2 & -- \\
		gute Bedingungen      & 3 & gutes Werkzeug \\
		durchschnittlich      & 4 & -- \\
		schlechte Bedingungen & 5 & wenig Licht \\
		schwer                & 6 & Messer jonglieren \\
		meisterlich           & 8 & Drahtseilakt \\
		legendär              & 12 & -- \\
	}
}

\renewcommand{\textPlayer}{%
	\noindent
	Jeder \keyword{Charakter} startet als unbeschriebenes A4-Blatt im Querformat. Dieses \keyword{Charakterblatt} wird durch eine Linie in zwei A5-Hälften geteilt, dann die rechte Hälfte in zwei A6-Viertel.

	Nun kritzeln die Spieler allgemeine Charakteristika wie Name, Volk und Aussehen in die linke Hälfte, gefolgt von der \keyword{Vorgeschichte} des Charakters. Dies kann in Stichworten oder Prosa geschehen. Die Beschreibung enthält, was ein Charakter bisher gemacht hat, und nicht, was er gut kann. Letzteres wird der Spielleiter erst im Spiel anhand der Vorgeschichte entscheiden. Am Charakterblatt steht dann \zB\ \say{war jahrelang Klavierträger} statt \say{ist stark}. \keyword{Ausrüstung} (\refPage{labelEquipment}) und die beherrschten \keyword{Effekte} (\refPage{labelEffects}) enthalten, was Spieler und Spielleiter für richtig befinden.

	Zuletzt werden je ein W4, W6, W8, W10 und W12 auf das rechte obere Viertel gelegt. Einer davon wird vom Spieler zum \keyword{Widerstandswürfel} (\HD) ernannt und mit seiner höchsten Zahl nach oben in die linke Blatthälfte gelegt. Sinkt sein Wert im Spiel unter 1, scheidet der Charakter aus.

	\mysection[labelTaskresolution]{Würfelsystem}

	\noindent
	Solange es keine Zweifel gibt, blubbert die Handlung zwischen Spielern und Spielleiter fröhlich vor sich hin. Stellt sich aber bei einer handlungsrelevanten \keyword{Aktion} die Frage, ob sie einem Charakter gelingt, wählt dessen Spieler einen \keyword{verfügbaren Würfel} aus dem oberen Viertel des Charakterblatts und würfelt. Fällt eine Eins, ist die Aktion ein \keyword{automatischer Fehlschlag}. Ansonsten wird ein, von der \keyword{Expertise} des Charakters abhängiger \keyword{Modifikator} zum Wurf addiert, den der Spielleiter anhand seiner Vorgeschichte festlegt.

	\tableModifier

	\noindent
	Erreicht oder übertrifft das Ergebnis den vom Spielleiter vorgegebenen \keyword{Zielwert} (\TN) der Aktion, gelingt diese. Eine errechnete Eins ist im Gegensatz zur gewürfelten Eins kein automatischer Fehlschlag, aber selten hoch genug.

	\tableTargets

	\noindent
	Nach dem Wurf wird der nun \keyword{verbrauchte Würfel} in das untere Viertel des Charakterblatts gelegt. Erst wenn \emph{alle} Würfel verbraucht sind -- und nur dann! -- kann ein Spieler seinen Charakter \keyword{durchatmen} lassen, ehe die Würfel wieder nach oben gelegt werden: Er muss kurz -- während eines Konflikts eine komplette Runde -- aussetzen.

	Wird für einen Wurf ein neuer Modifikator festgelegt, sollte dieser auf dem Charakterblatt notiert werden, damit er nicht mehrfach bestimmt werden muss.

	Würfel, die nach Berücksichtigung der Expertise keine Chance auf einen Erfolg haben, dürfen nicht benutzt bzw. verbraucht werden. Möchte ein Charakter die misslungene Aktion eines Kollegen \keyword{wiederholen}, erhöht sich der Zielwert mit jedem Versuch um 1.

	\mysection[labelConflict]{Konflikte}

	\noindent
	Interessenskonflikte werden in \keyword{Runden} abgehalten, deren Länge der Spielleiter festlegt. Jede Runde darf jeder Charakter eine Aktion durchführen, \zB\ angreifen, und auf alle Aktionen seiner Gegner reagieren, \zB\ parieren. Die \HD\ der Gegner sind zu überwinden, egal ob mit oder ohne Gewalt.

	Zu Beginn jeder Runde wählen alle Spieler zeitgleich je einen \keyword{Aktionswürfel} (\AD) und einen \keyword{Reaktionswürfel} (\RD) aus ihren verfügbaren Würfeln. Mit diesen bestreiten sie diese Runde alle Aktionen bzw. Reaktionen. Freiwillig, oder wenn zu wenig Würfel verfügbar sind, kann auf den \AD\ oder \RD\ und damit auf Aktionen bzw. Reaktionen verzichtet werden. Wer \emph{keine} verfügbaren Würfel mehr hat, muss aussetzen und darf durchatmen.

	Der gewählte \AD\ gibt auch die \keyword{Initiative} in der Runde an. Kleine Würfel handeln vor großen (\zB\ W6 vor W8). Bei Gleichstand würfeln die Betroffenen die Initiative aus. Überraschte Charaktere dürfen in der ersten Runde keinen \AD\ wählen.

	Für einen erfolgreichen Angriff muss der \AD\ des Angreifers ein höheres Ergebnis aufweisen als der \RD\ des Gegners, eine \emph{gewürfelte} Eins ist immer ein Fehlschlag. Bei einem erfolgreichen Angriff erleidet der Reagierende eine \keyword{Wunde} und reduziert die Zahl am \HD\ um 1. Auch gewaltlose Aktionen helfen, den Widerstand der Gegner zu brechen: Sie verursachen \keyword{Traumata}, die durch Spielsteine o.\,Ä. symbolisiert werden. Sinkt der \HD\ unter 1, oder erreicht die Summe der Traumata den aktuellen \HD-Wert, ist der Gegner überwunden (eingeschüchtert, irritiert,~\ldots).

	Ein Charakter kann Gegner im Ausmaß seines halben \AD\ (zwei beim W4, drei beim W6,~\ldots) mit einem Wurf als \keyword{Flächenangriff} attackieren (\zB\ Rundumschlag, Gruppen einschüchtern, Feuerball,~\ldots), wenn er diese Runde keinen \RD\ gewählt hat. Der Wurf wird pro zusätzlichem Ziel um \makebox{-2} modifiziert. Jeder Gegner darf individuell reagieren.

	Nahkämpfer können einen \keyword{Mehrfachangriff} gegen einen einzelnen Gegner versuchen (\zB\ mehrere Hiebe/Schläge). Dies wird wie ein Flächenangriff gegen ein einzelnes Ziel gehandhabt, das gegen jeden Treffer individuell reagieren darf.

	Letztlich kann ein Charakter andere \keyword{decken}, wenn er diese Runde keinen \AD\ wählt. Er darf dafür die Angriffe von Gegnern in Reichweite im Ausmaß seines halben \RD\ auf sich umleiten. Misslingt dem Deckenden eine solche Reaktion, bekommt er selbst die Wunden.

	\mysection[labelEquipment]{Ausrüstung}

	\noindent
	Es gibt keine Ausrüstungstabelle. Normale Waffen verursachen grundsätzlich eine Wunde pro Treffer, besondere oder magische Waffen zwei und Feuer- oder Explosivwaffen drei bis vier. Improvisierte Ausrüstung bedingt \makebox{-1} auf den \AD. Rüstungen geben je nach Ausführung +1 oder +2 auf den \RD.

	\mysection[labelHeal]{Heilung}

	\noindent
	Nach jeder \keyword{Nachtruhe} wird durchgeatmet. Zudem kann versucht werden, den \HD-Wert zu erhöhen. Der Spieler merkt sich den Wert und würfelt den \HD. Wird der Wert überwürfelt, verheilt eine Wunde und der \HD\ wird um Eins höher wieder hingelegt, sonst mit dem ursprünglichen Wert.

	\keyword{Traumata} verheilen abhängig von ihrem Ursprung nach Maßgabe des Spielleiters. Einschüchterungsversuche klingen schon am Ende der jeweiligen Szene wieder ab, Ängste, Flüche, etc. schleppen die Charaktere manchmal tage- oder wochenlang mit.
}

\renewcommand{\textEffects}{\zlabel{labelEffects}%
	\noindent
	Von Charakteren beherrschte Magie, Wunder, PSI-Kräfte, Superkräfte u.ä. werden \keyword{Effekte} genannt und bereits bei der Erschaffung festgelegt. Sie sind, so überhaupt im Szenario erlaubt, auch dort geregelt, berufen sich aber u.\,U. auf folgendes \nipajin-Standardsystem:

	Eine \keyword{Vorbereitungszeit} (\PT) lang murmelt oder gestikuliert der Charakter. Ist sie \keyword{variabel}, wird sie nach Bedarf vom Spieler festgelegt. Es folgt der Wurf, ggf. mit den üblichen \makebox{-2} pro Zusatzziel bei Flächenangriffen. Dem Opfer steht ein Reaktionswurf zu, um dem Effekt vollständig zu entgehen. Gegenstände und opferlose Effekte haben einen vom Spielleiter festgelegten Zielwert. Bei Gelingen hält der Effekt an, solange sich der Charakter konzentriert, zzgl. einer \keyword{Nachwirkzeit} (\FT).

	\keyword{Nahkampfeffekte} verwunden wie Nah\-kampf\-waf\-fen, \zB\ \emph{Frosthand} oder \emph{Geisterschwert}. Ein Treffer verursacht eine Wunde. \PT:~1~Runde; \FT:~$\infty$

	\keyword{Fernkampfeffekte} verwunden wie Fern\-kampf\-waf\-fen und verbrauchen pro Anwendung eine limitierte Ressource, \zB\ benötigt \emph{Magisches Geschoß} Pulver aus dem Gürtelbeutel, \emph{Feuerball} eine Art magische Handgranate. \PT:~1~Runde; \FT:~$\infty$

	\keyword{K.\,O.-Effekte} machen Wesen handlungsunfähig, \zB\ \emph{Schlaf}, \emph{Versteinerung}, \emph{Angst} oder \emph{Bannen}. \PT:~variabel; \FT:~aufgewandte \PT

	\keyword{Unterstützungen} helfen einem Wesen oder verbessern einen Gegenstand in einem Aspekt, \zB\ \emph{Feuerresistenz}, \emph{Federfall}, \emph{Barriere} oder \emph{Licht}. \PT:~variabel; \FT:~aufgewandte \PT

	\keyword{Veränderungen} verformen oder bewegen langsam tote Materie bzw. Gefühle von Wesen, \zB\ \emph{Wasser-zu-Wein}, \emph{Befreunden} oder \emph{Telekinese}. \PT:~variabel; \FT:~aufgewandte \PT

	\keyword{Illusionen} täuschen einen Sinn eines Wesens für ein konkretes Detail, \zB\ \emph{Katzengold}, \emph{Täuschgeräusch} oder \emph{Unsichtbarkeit}. \PT:~variabel; \FT:~aufgewandte \PT

	\keyword{Eingebungen} verschaffen Wissen über Eigenschaften oder Sachverhalte, \zB\ \emph{Magie spüren}, \emph{Gedanken lesen} oder \emph{Hellsicht}. \PT:~eine Minute für Gegenwärtiges, eine Stunde für Vergangenes, ein Tag für Zukünftiges; \FT:~--

	\keyword{Heileffekte} schließen Wunden oder heilen Krankheiten. \PT:~eine Stunde pro Wunde, ein Tag pro Krankheit; \FT:~$\infty$
}

\renewcommand{\textGM}{%
	\mysection[labelPCs]{Vorgeschichten}

	\noindent
	Den Vorgeschichten der Spielercharaktere (SC) kommt in \nipajin\ besondere Bedeutung zu. Gute Hintergründe umschreiben Kindheit, Ausbildung und was ein SC in den letzten Jahren hauptsächlich getan hat. Ein paar einschneidende Erlebnisse runden das Bild ab. Auch Alter und Aussehen eines SC sollten festgehalten werden.

	Es liegt am jeweiligen Spielstil, wie formell eine Vorgeschichte ausfallen muss. Der Spielleiter sollte darauf achten, dass er genügend Rückschlüsse auf die Stärken und Schwächen des SC ermöglicht, da im Zweifel auf dieser Basis entschieden wird, ob eine Aktion leicht- oder schwerfällt. Lücken in der Vorgeschichte sollten im Einverständnis mit dem Spieler sofort geschlossen werden. Spielfertige Szenarien geben für Beispielcharaktere meist erste Vor- und Nachteile an -- diese dürfen im Spiel natürlich ergänzt werden.

	\mysection[labelGroups]{Gruppenarbeit}

	\noindent
	Arbeiten bei einer Aktion mehrere -- nicht notwendiger Weise alle -- SC zusammen, kommt es zu einer \keyword{Gruppenaktion}. Die beteiligten Spieler wählen zuerst je ihren \AD\ und ernennen danach einen der SC zum Anführer. Dessen Spieler würfelt als Erster und entscheidet, ob das Ergebnis stellvertretend für alle zählt. Wenn nicht, übergibt er die Führung an einen verbleibenden SC, usw. Jeder Wurf \emph{ersetzt} den vorherigen. Eine gewürfelte Eins vereitelt die Gruppenaktion für alle und beendet diese. Nur die tatsächlich benutzten Würfel werden verbraucht.

	Bei einer Gruppenaktion innerhalb eines Konflikts gilt der schlechteste Initiativewert für die ganze Gruppe. Der Gegner tritt mit seinem \RD\ gegen das Ergebnis an, ihm droht die Summe der Wunden, die alle Beteiligten einzeln verursachen würden. Gruppenaktionen lassen sich nicht mit Flächen- oder Mehrfachangriffen kombinieren.

	\keyword{Langfristige Aktionen} haben hohe Zielwerte, \zB\ Reparatur\TN20. Die beteiligten SC arbeiten in mehreren Etappen -- einzelne Gruppenaktionen, innerhalb derer nur \emph{ersetzt} wird -- auf diese Summe hin. Der SL legt fest, wieviel Zeit pro Etappe vergeht, z.B. ein Tag. Eine gewürfelte Eins vereitelt die Etappe, aber nicht die langfristige Aktion.

	\mysection[labelNSCs]{Nichtspielercharaktere}

	\noindent
	Der Spielleiter wird den SC eine Reihe von Freunden und Feinden entgegensenden. Diese \keyword{Nichtspielercharaktere} (NSC) werden vereinfacht geregelt: Neben dem Aussehen und der Motivation, mit den SC zu interagieren, genügt es, pauschal den \HD\ sowie die Grundkompetenz in Form \emph{eines} \AD\ und \emph{eines} \RD\ festzulegen. Hier sind auch Abstufungen wie W2 oder W3 erlaubt.

	\tabelle{c X}{
	\thead{\AD/\RD} & \thead{Grundkompetenz}  \\
	}{
		W2 & nur in Gruppen gefährlich \\
		W3 & blutiger Anfänger \\
		W4 & besserer Anfänger \\
		W6 & durchschnittlich \\
		W8 & routiniert \\
		W10 & gefährlich \\
		W12 & sehr gefährlich \\
		W20 & episch \\
	}

	\noindent
	Daneben werden ein paar Stärken und Schwächen notiert, \zB\ Kämpfen+1 oder Geschick-2. Dabei ist zu beachten, dass NSC nicht unabsichtlich doppelt gut werden, weil sie hohe Grundkompetenz \emph{und} eine Stärke erhalten.

	NSC haben gegenüber SC den Vorteil, nicht durchatmen zu müssen, da ihnen die fixen \AD/\RD\ nie ausgehen. Sie sollten zum Ausgleich etwas unterdimensioniert werden. Weiters haben Traumata für sie selten langfristige Bedeutung und werden direkt am \HD\ mitgezählt.

	\mysection[labelBestiary]{Bestiarium}

	\noindent
	Die folgenden Kreaturen dienen nur der Veranschaulichung und sollen \nipajin\ nicht auf bestimmte Hintergründe festlegen.

	\tabelle{p{1.1cm} c c c X}{
	\thead{Kreatur} & \thead{\HD} & \thead{\AD} & \thead{\RD} & \thead{Fähigkeiten}  \\
	}{
		gr. Ratte & 1 & 2 & 3 & Laufen+4, Verstecken+2 \\
		Goblin & 3 & 4 & 4 & Wahrnehmung+1 \\
		Ork & 6 & 6 & 6 & Einschüchtern+1, Kämpfen+1, Verstand-1 \\
		Troll & 10 & 8 & 6 & Kämpfen+1, regeneriert eine Wunde/Runde \\
		Riese & 20 & 8 & 8 & Kämpfen+2, Kraft+4 \\
		Drache & 40 & 12 & 10 & Feueratem+4, K.\,O.-Effekt-Resistenz \\
	}

	\mysection[labelXP]{Erfahrung}

	\noindent
	Die SC definieren das Niveau ihrer Umwelt. Verkörpern sie durchschnittliche Abenteurer, sind die im Bestiarium angegebenen Beispiele gute Richtwerte für \say{die Anderen}. Verkörpern sie Goblins, die sich Scharen von \say{Helden} erwehren müssen, die über ihr Lager herfallen, sind die \say{Helden} für die Goblins vielleicht schon so mächtig wie ein Troll. Der Spielleiter sollte SC und NSC relativ zueinander betrachten.

	\nipajin\ ist nicht darauf ausgelegt, Charaktere über lange Kampagnen wachsen zu sehen. Sollte allerdings während eines Szenarios in der Spielwelt genug Zeit mit Training verbracht werden, kann der Spielleiter eine bereits festgelegte Expertise eines SC erhöhen, wenn dies plausibel erscheint. Sollten die SC über Nacht an Super\-hel\-den\-kräfte gelangen, empfiehlt es sich, \emph{nicht die Charaktere} zu steigern, \emph{sondern die Welt} um sie herum entsprechend abzusenken!
}
